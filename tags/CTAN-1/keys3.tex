%%%%%%%%%%%%%%%%%%%%%%%%%%%%%%%%%%%%%%%%%%%%%%%%%%%%%%%%%%%%%%%%%%%%%%%
% This is a demonstration file for keys3, an experimental package for
% key management in LaTeX3.  The file should compile giving two
% \show results and exactly six errors:
% - "The key `/module name/key seven' requires a value and is being
%   ignored."
% - "The key `/module name/key eight' cannot taken a value, which is
%   being ignored."
% - "Key `/module name/key ten' has multiple arguments and was
%   expanded: adding code may give incorrect results."
% - "Key `/module name /key eighteen' takes the Boolean values `true'
%   and `false' only. The given value `rubbish' is ignored."
% - "Choice `what?' unknown for key `/module name/key twenty': the
%   value given is being ignored."
% - "The key `/module name/unknown key' is unknown and is being
%   ignored."
%%%%%%%%%%%%%%%%%%%%%%%%%%%%%%%%%%%%%%%%%%%%%%%%%%%%%%%%%%%%%%%%%%%%%%%

\documentclass{article}
\usepackage{keys3,xparse}

% The document commands for creating keys is very simple. A generic
% one is created to set any modules keys, and one specific to
% the module used here.
\ExplSyntaxOn
\DeclareDocumentCommand{\SetModuleKeys}{m}{
  \keys_manage:n{
    module~name/.cd:,
    #1}
}

\DeclareDocumentCommand{\SetKeys}{m}{
  \keys_manage:n{#1}
}

% A tlp for demonstration purposes.
\tlp_new:N \tlpone
\tlp_set:Nn \tlpone {TLP~Contents}

\ExplSyntaxOff

\begin{document}

% Everything is done internally using \cs:w ... \cs_end:, and so you
% can create keys without needing \CodeStart ... \CodeStop

% The first key is very simple, it just prints the users input. Notice
% that the surrounding spaces are removed, but the internal space is
% retained.
\SetModuleKeys{
  key one/.code:n = {You said `#1'},
  key one = Hello World ,
}

% Next, a key with arbitrary parameter format is created.
\SetModuleKeys{
  key two/.code:nn = {#1 to #2} {You said `#1' to `#2'},
  key two = Hello to you ,
}

% The package automatically tries the root path, so the following
% also works.
\SetKeys{
  module name/key one = Something else
}


% A module can be declared using the ".module:" property. The
% module name then acts as a ".cd:" property.
\SetKeys{
  module name/.module:,
  module name,
  key one = Something else again
}

% A couple of keys created using the :x rather than :n method to
% fully-expand the argument.
\newcommand*{\demo}{ABC}
\SetModuleKeys{
  key three/.code:x = {You said `#1' to `\demo'},
  key four/.code:nx = {#1 or #2}
    {You said `#1~\demo', or perhaps `#2'},
}
\renewcommand*{\demo}{DEF}
\SetModuleKeys{
  key three = Hello
}

\SetModuleKeys{
  key four = Goodbye or whatever
}

% Keys with initial and default values are created. With an initial
% value, the key will contain something even if never used. This
% is overwritten is something else is then done with the key, such as
% assigning code. In contrast, a default value is used when a key is
% given with no value, but is not assigned until the key is first
% encountered. The default property still requires some code to be
% defined.
\SetModuleKeys{
  key five/.initial:n = Stuff,
  key six/.code:n = { or perhaps #1},
  key six/.default:n = more stuff,
  key five,
  key five/.code:n = #1,
  key five, % Nothing appears for this key now
  key six,
  key six = even more stuff!
}

% Values can be forbidden or required: both of the below should give an
% error.
\SetModuleKeys{
  key seven/.code:n = (#1),
  key eight/.code:n = Hello,
  key seven/.value required:,
  key eight/.value forbidden:,
  key seven,
  key eight = Stuff,
}

% Code can be added to the left or right of existing keys. The added
% code can include #1, etc. as appropriate to the original definition.
% At present, code added to a key generated with .code:nx may end up
% with a defective definition. So an error is raised if that happens,
% when an attempt is made to use the key.
\renewcommand*{\demo}{ABC}
\SetModuleKeys{
  key nine/.code:nn = {#1#2}{#1 or #2},
  key nine/.put code right:n = { and so on #2},
  key nine = {Words}{more words},
  key ten/.code:nx = {#1#2}{\demo\ #1 again #2},
}
\renewcommand*{\demo}{DEF}
\SetModuleKeys{
  key ten/.put code left:n = {You said: },
  key ten = {things}{of course}
}

% Keys can be used to set other keys.  The various .set keys:
% properties achieve this. The .set keys:nn and .set keys:nx
% properties have two arguments, as they require the usual
% parameter format.
\SetModuleKeys{
  key eleven/.keys:n = {key one = #1},
  key eleven = Things
}

% Watch the spacing with multiple argument: you cannot have any spaces,
% as it confuses the underlying system.
\SetModuleKeys{
  key twelve/.keys:nn =
    {#1#2}{key one = {#1 and also #2}},
  key twelve = {Things}{more things} % No space here
}

% Keys can be set to tlp values, or vice versa.
\SetModuleKeys{
  key thirteen/.set equal:N = \tlpone,
  key thirteen,
  key fourteen/.initial:n = New contents,
  key fourteen/.set tlp:N = \tlpone
}
\tlpone

% Adding to a stored value.
\SetModuleKeys{
  key fifteen/.set equal:N = \tlpone,
  key fifteen/.put right:n = { extended},
  key fifteen,
  key fifteen/.put left:n = {See how },
  key fifteen
}

% Data can be stored directly in a tlp, either as given or expanded.
\renewcommand*{\demo}{ABC}
\SetModuleKeys{
  key sixteen/.set:N = \tlpone,
  key sixteen = Text
}
\tlpone
\SetModuleKeys{
  key seventeen/.set exp:N = \tlpone,
  key seventeen = \demo
}
\renewcommand*{\demo}{DEF}
\tlpone

% Boolean keys can take "true" and "false" only, with a default value
% of "true". The Boolean is created by the package.
\SetModuleKeys{
  key eighteen/.boolean:N = \test,
  key eighteen = true,
  key eighteen = false,
  key eighteen,
  key eighteen = rubbish,
  key eighteen/.complement:n = key nineteen,
  key nineteen = true
}

% Choices are created as sub-keys of the choice.
\SetModuleKeys{
  key twenty/.choice:,
  key twenty/choice one/.code:n = One,
  key twenty/choice two/.code:n = Two,
  key twenty/choice three/.code:n = Three,
  key twenty/choice three/.equivalent:n = choice four,
  key twenty = choice one,
  key twenty = choice four,
  key twenty = what?
}

% Similar choices
\SetModuleKeys{
  key twenty-one/.choices:nn =
    {fish,chips,egg}{You said \csname l_keys_current_choice_tlp\endcsname},
  key twenty-one/fish,
  key twenty-one/chips,
}

% Some expansion control
\newcommand*{\aaa}{bottom}
\newcommand*{\bbb}{\aaa}
\newcommand*{\ccc}{\bbb}
\SetModuleKeys{
  key twenty-two/.initial:n=\ccc,
  key twenty-three/.initial:n/.expand:o=\ccc,
  key twenty-four/.initial:n/.expand:d=\ccc,
  key twenty-five/.initial:n/.expand:x=\ccc
}
\renewcommand*{\aaa}{a }
\renewcommand*{\bbb}{b }
\renewcommand*{\ccc}{c }
\SetModuleKeys{
  key twenty-two,
  key twenty-three,
  key twenty-four,
  key twenty-five
}

%
\SetModuleKeys{
  key twenty-six/.code:n = (#1),
  key twenty-six/.try:n = Look out,
  key twenty-six/.try:n = hmm,
  key twenty-six/.try:n = Llamas!,
  key twenty-six/.retry:n = Badgers,
}
\SetModuleKeys{
  key twenty-six/.code:n = (#1),
  key twenty-six/.retry:n = Look out,
  key twenty-six/.try:n = hmm,
  key twenty-six/.retry:n = Llamas!,
  key twenty-six/.retry:n = Badgers,
}

% Keys can be debugged using the ".show key:" and ".show code:"
% properties.
\SetModuleKeys{
  key fifteen/.show key:,
  key one/.show code:
}

% Code can be run directly
\SetModuleKeys{
  /keys/execute:n = Hello World!
}

% This key has never been defined, and so the next call gives an error.
\SetModuleKeys{
  unknown key = value}

\end{document}
