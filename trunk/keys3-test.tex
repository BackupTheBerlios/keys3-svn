\immediate\write18{tex keys3.ins}

\documentclass{article}
\usepackage{keys3,xparse}
\usepackage[parfill]{parskip}

% The document commands for creating keys is very simple. A generic
% one is created to set any modules keys, and one specific to
% the module used here.
\ExplSyntaxOn
\DeclareDocumentCommand{\SetModuleKeys}{m}{
  \keys_manage:n{
    module~name/.cd:,
    #1}
}

\DeclareDocumentCommand{\SetKeys}{m}{
  \keys_manage:n{#1}
}

% A tlp for demonstration purposes.
\tlp_new:N \tlpone
\tlp_set:Nn \tlpone {TLP~Contents}

\ExplSyntaxOff

\begin{document}

% Test one: Simply creating a key with one argument and then using it.
% Notice that the text length determines the limits of the argument: 
% spaces at either end are ignored. Literal spaces between text are 
% counted, as are those enclosed in braces.
% ----------------------------------------------------------------------
%   You said: `Hello World'
%   You said: ` Hello World '
%   You said: `'
% ----------------------------------------------------------------------
\SetModuleKeys{
  key/.code:n = {You said: `#1'\\},
  key = Hello World ,
  key = { Hello World },
  key = 
}

% Test two: Creating a multiple-argument key. Here, the arguments
% given need to be at least as many as those required by the code!
% ----------------------------------------------------------------------
%   You said: `A', `B', `C', D'.
%   You said: `AAA', `BBB', `CCC', ` DDD'.
%   You said: `', `', `', `'.
% ----------------------------------------------------------------------
\SetModuleKeys{
  key/.code:Nn = 4 {You said: `#1', `#2', `#3', `#4'.\\},
  key = ABCD,
  key = {AAA}{BBB} {CCC} { DDD},
  key =,
}
\end{document}
