% \iffalse meta-comment
% !TEX program  = pdflatex
% !TEX encoding = ISO-8859-1
%<*internal>
\iffalse
%</internal>
%<*readme>
---------------------------------------------------------------
The keys3 package --- Key management for LaTeX3
Maintained by Joseph Wright 
E-mail: joseph.wright@morningstar2.co.uk
Released under the LaTeX Project Public License v1.3c or later
See http://www.latex-project.org/lppl.txt
---------------------------------------------------------------

The keys3 package is an experimental implementation of key
management for LaTeX3.  The entire package is written in
expl3 code rather than in traditional TeX/LaTeX.  The key
module used by keys3 is based on that introduced by the
pgfkeys package from the pgf/TiKZ bundle. Keys are thus
addressed in a path-like manner, and are controlled by key
properties:

/module/key/.code:n = <code>,
/module/key/.value_required:,
/module/key = <value>

The keys3 package is intended as a method for testing how key
management might be implemented in LaTeX3.  Both the methods
used inside the package and the interface provided are intended
to be discussed.  Feedback is welcomed either by e-mail
(joseph.wright@morningstar2.co.uk) or at the package website
(http://keys3.berlios.de).

The experimental nature of keys3 and the underlying expl3
system means that it should not be used in production
documents.  However, programmers testing expl3, or keen to work
on improved key management systems, are encouraged to try the
package and provide feedback.

keys3 is *not* an official LaTeX3 team package.  Whether any of
the ideas explored here will be taken up by the team is
therefore unknown.  However, the keys3 package should provide
ideas on what needs to be available for key management and how
it can be implemented.
%</readme>
%<*internal>
\fi
\def\nameofplainTeX{plain}
\ifx\fmtname\nameofplainTeX\else
  \expandafter\begingroup
\fi
%</internal>
%<*install>
\input docstrip.tex
\keepsilent
\askforoverwritefalse
\preamble
---------------------------------------------------------------
The keys3 package --- Key management for LaTeX3
Maintained by Joseph Wright 
E-mail: joseph.wright@morningstar2.co.uk
Released under the LaTeX Project Public License v1.3c or later
See http://www.latex-project.org/lppl.txt
---------------------------------------------------------------

EXPERIMENTAL CODE

Do not distribute this file without also distributing the
source files specified above.

Do not distribute a modified version of this file.

\endpreamble
\postamble

Copyright (C) 2008-2009 by
  Joseph Wright <joseph.wright@morningstar2.co.uk>

It may be distributed and/or modified under the conditions of
the LaTeX Project Public License (LPPL), either version 1.3c of
this license or (at your option) any later version.  The latest
version of this license is in the file:

   http://www.latex-project.org/lppl.txt

This work is "maintained" (as per LPPL maintenance status) by
  Joseph Wright.

This work consists of the file  keys3.dtx
          and the derived files keys3.pdf,
                                keys3.sty and
                                keys3.ins.

\endpostamble
\usedir{tex/latex/keys3}
\generate{
  \file{\jobname.sty}{\from{\jobname.dtx}{package}}
}
%</install>
%<install>\endbatchfile
%<*internal>
\usedir{source/latex/keys3}
\generate{
  \file{\jobname.ins}{\from{\jobname.dtx}{install}}
}
\nopreamble\nopostamble
\usedir{doc/latex/keys3}
\generate{
  \file{README.txt}{\from{\jobname.dtx}{readme}}
}
\ifx\fmtname\nameofplainTeX
  \expandafter\endbatchfile
\else
  \expandafter\endgroup
\fi
%</internal>
%<*driver|package>
\RequirePackage{l3names}
\GetIdInfo$Id$
  {Key management for LaTeX3}
%</driver|package>
%\fi
%\iffalse
%<*driver>
%\fi
\ProvidesFile{\filename.\filenameext}
  [\filedate\space v\fileversion\space\filedescription]
%\iffalse
\documentclass[full]{l3doc}
\begin{document}
  \DocInput{\filename.\filenameext}
\end{document}
%</driver>
% \fi
%
% \title{The \pkg{keys3} package^^A
%   \thanks{This file has version number \fileversion, last
%     revised \filedate.}\\
% Key management for \LaTeX3}
% \author{Joseph Wright^^A
%   \thanks{E-mail: joseph.wright@morningstar2.co.uk}}
% \date{\filedate}
% \maketitle
%
%\section{Key management}
%
% The key--value method is a popular system for creating large numbers
% of settings for controlling macro or package behaviour.  For the
% user, the system normally results in input of the form
%\begin{verbatim}
%  \PackageControlMacro{
%    key     = value,
%    key two = value two
%  }
%\end{verbatim}
% or
%\begin{verbatim}
%  \PackageMacro[
%    key     = value,
%    key two = value two
%  ]{argument}.
%\end{verbatim}
% For the programmer, the original \pkg{keyval} package gives only
% the most basic interface for this work.  All key macros have to be
% created one at a time, and as a result the \pkg{kvoptions} and
% \pkg{xkeyval} packages have been written to extend the ease of
% creating keys.  However, the underlying model is rather inflexible.
%
% As an alternative, \pkg{pgfkeys} from the \pkg{pgf}/Ti\emph{k}z
% bundle uses a ``file-like'' model for keys. In this model, each key
% has one or more functions attached to define its action.  In
% \pkg{pgfkeys} terminology, these are key \emph{handlers}. Keys can
% be created and used using a unified interface:
%\begin{verbatim}
%  \pgfkeys{
%    /path/key/.handler = code,
%    /path/key          = value
%  }
%\end{verbatim}
%
% The \pkg{keys3} package is aimed at creating a high-level
% programmers interface for key--value controls in \LaTeX3. Key
% creation and control follows the \pkg{pgfkeys} closely, although
% changes have been made to adhere to the new coding ideas of \LaTeX3,
% and new functions have been added.  The \pkg{keys3} package is
% \emph{not} a simple translation of \pkg{pgfkeys} to the new syntax.
% In package internals have been written from the ground up, to better
% enforce variable typing and to act in a know manner under failing
% circumstances.
%
% In the \pkg{keys3} model, each key belongs to a module,
% which may have one or more subdivisions. Each key then has
% one or more properties which define how the key acts. The idea
% of a single function for all key control is retained:
%\begin{verbatim}
%  \keys_manage:n{
%    /module/key/.property     = code,
%    /module/sub/key/.property = more code,
%    /module/key               = value
%  }
%\end{verbatim}
% The combination of all \m{module} parts up to the \m{key} is referred
% to as the \emph{path} of the key.
%
%\subsection{Creating, retrieving and setting keys}
%
% The main interface for key management is the the \cs{keys_manage:n}
% macro.  This can be used to create, retrieve and set keys, and is
% therefore the preferred interface for the package.
%
% \begin{function}{
%   \keys_manage:n|
%   \keys_manage_quick:n|
%   \keys_manage_internal:n|
% }
%   \begin{syntax}
%     "\keys_manage:n" <keyval list>
%   \end{syntax}
%   The main key management function, which parses over the <keyval
%   list> and processes each key. The standard version removes leading
%   and trailing spaces and checks catcodes for ungrouped ``|,|'' and
%   ``|=|''. The |_quick| version omits these tests for speed, and is
%   therefore more suited to creating keys inside code blocks. The
%   \texttt{internal} variant is used when recycling keys inside
%   \pkg{keys3}.
% \end{function}
%
% \begin{function}{
%   \keys_manage:nn|
%   \keys_manage_quick:nn|
% }
%   \begin{syntax}
%     "\keys_manage:nn" <path> <keyval list>
%   \end{syntax}
%   Versions of the above which include a <path> as an argument. These
%   are faster versions of the |:n| macros when setting lots of keys
%   separately and with known paths.
% \end{function}
%
% Setting up and altering keys is carried out using one or more
% properties (in \pkg{pgfkeys}, these are called handlers). In all
% cases, \m{key} may be a full-qualified key with a path, or a partial
% key to which the default path will be added. Notice that when a
% single argument is required, the braces around the argument may be
% omitted without any error arising.
% 
%\subsubsection{Storing values}
%
% A common use of key--value input is to store the values given in a 
% variable for later use.  \pkg{keys3} provides pre-defined 
% properties for storing in int, skip, tl and toks variables. The 
% method is similar in all cases:
%\begin{verbatim}
%  \keys_manage:n{
%    /module/key~one/.int_set:N = \l_mod_data_int,
%    /module/key~two/.skip_set:N = \l_mod_data_skip,
%    /module/key~three/.tl_set:N = \l_mod_data_tl,
%    /module/key~four/.toks_set:N = \l_mod_data_toks
%  }
%\end{verbatim}
% Global variants for all of these are.
%   
% 
% When using keys to store a value, it is often convenient to give only
% the unique part of the variable name. This is particularly true when
% creating a large number of related storage areas. To achieve this,
% \pkg{keys3} requires that the module prefix to be used is defined 
% first. The special key \texttt{/keys/current_module:n} can be set with 
% the name of the current module.
%\begin{verbatim}
%  \keys_manage:n{
%    /keys/current_module:n = module,
%    /module/key/.tl_set:n = my_data,
%  }
%\end{verbatim}
% This will use a variable called \cs{l_module_my_data_tl} to store the
% input. Thus the preceding code achieves the same effect as
%\begin{verbatim}
%  \keys_manage:n{
%    /module/key/.tl_set:N = \l_module_my_data_tl
%  }
%\end{verbatim}
% The same method applies to int, skip and toks variables. The 
% appropriate prefix (\cs{l_} or \cs{g_}) and suffix (|_int|, |_skip|, 
% |_tl| or |_toks|) is always added.
% 
%\subsubsection{Multiple choice keys}
%
% Multiple choice keys are created in \pkg{keys3} using the 
% \texttt{.expects_choice:} property. Each choice is then a sub-key of 
% the choice key.
%\begin{verbatim}
%  \keys_manage:n{
%    /module/key/.expects_choice:,
%    /module/key/choice~a/.code:n = Some code,
%    /module/key/choice~b/.code:n = Some other code,
%  }
%\end{verbatim}
% In this way, choices which execute arbitrary code can be created. 
% Notice that the \m{code} should \emph{not} include a parameter (|#1|).
% 
% Often it is desirable to create a family of similar choices, which 
% only require either the text of the choice, or the position of the 
% choice in a list, to be used.  To create this type of simple choice,
% \pkg{keys3} provides the \texttt{.create_choices:nn} property. This
% applies the same code to a list of choice text. Inside the code,
% the name of the choice given is available as 
% \cs{l_keys_current_choice_tl}. The position of the choice in the 
% lists is also available, as \cs{l_keys_current_choice_int}. 
%\begin{verbatim}
%  \keys_manage:n {
%    /module/key/.create_choices:nn = {choice~a, choice~b, choice~c} {
%      You~gave~choice~``\l_keys_current_choice_tl'',~which~is~in~
%      position~\l_keys_current_choice_int~in~the~list.
%    }
%  }
%\end{verbatim}
% 
%\subsection{Properties}
%
% \begin{function}{
%   .bool_set:N|
%   .bool_gset:N|
% }
%  \begin{syntax}
%    <key>/.bool_set:N = <bool>
%  \end{syntax}
%  Defines <key> to set <bool> to <value> (which must be either 
%  \texttt{true} or \texttt{false}).
% \end{function}
% 
% \begin{function}{
%   .bool_set_inverse:n|
%   .bool_gset_inverse:n|
% }
%  \begin{syntax}
%    <key>/.bool_set_inverse:n = <name>
%  \end{syntax}
%  Defines <key> to set switch with unique <name> to <value> (which 
%  must be either \texttt{true} or \texttt{false}), with reversed logic. 
%  The switch name will be constructed using the current <module> (if 
%  any), prefixed by |\l_| and ending with |_int|.
% \end{function}
% 
% \begin{function}{
%   .bool_set_inverse:N|
%   .bool_gset_inverse:N|
% }
%  \begin{syntax}
%    <key>/.bool_set_inverse:N = <bool>
%  \end{syntax}
%  Defines <key> to set <bool> to <value> (which must be either 
%  \texttt{true} or \texttt{false}), with reversed logic.
% \end{function}
% 
% \begin{function}{
%   .bool_set:n|
%   .bool_gset:n|
% }
%  \begin{syntax}
%    <key>/.bool_set:n = <name>
%  \end{syntax}
%  Defines <key> to set switch with unique <name> to <value> (which 
%  must be either \texttt{true} or \texttt{false}). The switch
%  name will be constructed using the current <module> (if any), 
%  prefixed by |\l_| and ending with |_int|.
% \end{function}
%
% \begin{function}{.cd:}
%  \begin{syntax}
%    <key>/.cd:
%  \end{syntax}
%  Changes path to that given by <key>.
% \end{function}
%
% \begin{function}{
%   .code:n|
%   .code:x
%  }
%  \begin{syntax}
%    <key>/.code:n = <code>
%  \end{syntax}
%  Stores the <code> for execution when <key> is called. The <code> can
%  include one parameter (|#1|).
% \end{function}
% 
% \begin{function}{
%   .code:Nn|
%   .code:Nx
% }
%  \begin{syntax}
%    <key>/.code:Nn = <number> <code>
%  \end{syntax}
%  Stores the <code> for execution when <key> is called. The <code> can
%  include <number> parameters, which can be in the range 0--9.
% \end{function}
% 
% \begin{function}{
%   .create_choices:nn|
%   .create_choices:nx|
% }
%  \begin{syntax}
%    <key>/.create_choices:nn = <list> <code>
%  \end{syntax}
%  Creates a sub-key of <key> for each <choice> in the comma-separated
%  <list>. Each <choice> will have <code> associated with it. The 
%  current choice text is available as \cs{l_keys_current_choice_tl},
%  and its position in the <list> as \cs{l_keys_current_choice_int}.
% \end{function}
% 
% \begin{function}{.default:n}
%  \begin{syntax}
%    <key>/.default:n = <content>
%  \end{syntax}
%  Creates a default value for <key>, which is used if no value is 
%  given. The <content> is stored as a tl, and so must be compatible
%  with this variable type.
% \end{function}
% 
% \begin{function}{.expects_choice:}
%  \begin{syntax}
%    <key>/.expects_choice:
%  \end{syntax}
%  Indicates that the <value> given for <key> should be a sub-key of
%  <key>. In this way, <key> accepts one of a limited range of choices.
% \end{function}
% 
% \begin{function}{
%   .int_set:N|
%   .int_gset:N|
% }
%  \begin{syntax}
%    <key>/.int_set:N = <int>
%  \end{syntax}
%  Defines <key> to store <value> in the <int> named.
% \end{function}
% 
% \begin{function}{
%   .int_set:n|
%   .int_gset:n|
% }
%  \begin{syntax}
%    <key>/.int_set:n = <name>
%  \end{syntax}
%  Defines <key> to store <value> in a int with unique name <name>. The
%  int name will be constructed using the current <module> (if any), 
%  prefixed by |\l_| and ending with |_int|.
% \end{function}
% 
% \begin{function}{.retry:n}
%  \begin{syntax}
%    <key>/.retry:n = <data>
%  \end{syntax}
%  Executes <key> if it exists and if the previous \texttt{.try:n} 
%  failed. The <data> is passed to <key>, if successful.
% \end{function}
% 
% \begin{function}{
%   .show_code:|
%   .show_key:|
% }
%  \begin{syntax}
%    <key>/.show_code:
%    <key>/.show_key:
%  \end{syntax}
%  Shows the function for the current key or the code associated with
%  the current key.
% \end{function}
% 
% \begin{function}{
%   .skip_set:N|
%   .skip_gset:N|
% }
%  \begin{syntax}
%    <key>/.skip_set:N = <skip>
%  \end{syntax}
%  Defines <key> to store <value> in the <skip> named.
% \end{function}
% 
% \begin{function}{
%   .skip_set:n|
%   .skip_gset:n|
% }
%  \begin{syntax}
%    <key>/.skip_set:n = <name>
%  \end{syntax}
%  Defines <key> to store <value> in a skip with unique name <name>. The
%  skip name will be constructed using the current <module> (if any), 
%  prefixed by |\l_| and ending with |_skip|.
% \end{function}
% 
% \begin{function}{
%   .tl_set:N|
%   .tl_set_x:N|
%   .tl_gset:N|
%   .tl_gset_x:N|
% }
%  \begin{syntax}
%    <key>/.tl_set:N = <tl>
%  \end{syntax}
%  Defines <key> to store <value> in the <tl> named. The \texttt{x}
%  versions use \cs{tl_(g)set:Nx} for this process, the standard 
%  version \cs{tl_(g)set:Nn}.
% \end{function}
% 
% \begin{function}{
%   .tl_set:n|
%   .tl_set_x:n|
%   .tl_gset:n|
%   .tl_gset_x:n|
% }
%  \begin{syntax}
%    <key>/.tl_set:n = <name>
%  \end{syntax}
%  Defines <key> to store <value> in a tl with unique name <name>. The
%  tl name will be constructed using the current <module> (if any), 
%  prefixed by |\l_| and ending with |_tl|.
% \end{function}
% 
% \begin{function}{
%   .toks_set:N|
%   .toks_gset:N|
% }
%  \begin{syntax}
%    <key>/.toks_set:N = <toks>
%  \end{syntax}
%  Defines <key> to store <value> in the <toks> named.
% \end{function}
% 
% \begin{function}{
%   .toks_set:n|
%   .toks_gset:n|
% }
%  \begin{syntax}
%    <key>/.toks_set:n = <name>
%  \end{syntax}
%  Defines <key> to store <value> in a toks with unique name <name>. The
%  rest of the toks name will be constructed using the current <module>
%  (if any), prefixes by |\l_| and ending with |_toks|.
% \end{function}
% 
% \begin{function}{.try:n}
%  \begin{syntax}
%    <key>/.try:n = <data>
%  \end{syntax}
%  Executes <key> if defined, and does nothing otherwise.  The <data> is
%  passed to <key>, if successful.
% \end{function}
% 
% \begin{function}{
%   .use_keys:n|
%   .use_keys:x|
% }
%  \begin{syntax}
%    <key>/.use_keys:n = <keyval list>
%  \end{syntax}
%  Calling <key> applies the <keyval list> as a block. Thus one <key> 
%  can make many related settings. As usual, |#1| is available in the
%  <keyval list> as <key> will take one argument.
% \end{function}
% 
% \begin{function}{
%   .use_keys:Nn|
%   .use_keys:Nx|
% }
%  \begin{syntax}
%    <key>/.use_keys:Nn = <number> <keyval list>
%  \end{syntax}
%  Calling <key> applies the <keyval list> as a block. Thus one <key> 
%  can make many related settings. The <key> will take <number> of 
%  arguments, which can be used inside the <keyval list>.
% \end{function}
% 
% \begin{function}{
%   .value_forbidden:|
%   .value_required:|
% }
%  \begin{syntax}
%    <key>/.value_forbidden:
%  \end{syntax}
%  Flags for forbidding and requiring a <value> for <key>.
% \end{function}
% 
%\subsection{Variables and constants}
%
% \begin{variable}{
%   \c_keys_0_empty_tl|
%   \c_keys_1_empty_tl|
%   \c_keys_2_empty_tl|
%   \c_keys_3_empty_tl|
%   \c_keys_4_empty_tl|
%   \c_keys_5_empty_tl|
%   \c_keys_6_empty_tl|
%   \c_keys_7_empty_tl|
%   \c_keys_8_empty_tl|
%   \c_keys_9_empty_tl|
% }
%   A set of tls containing empty groups.
% \end{variable}
% 
% \begin{variable}{\c_keys_cs_prefix_tl}
%   The prefix added to the fully-qualified key when saving them.
% \end{variable}
% 
% \begin{variable}{
%   \c_keys_errors_path_tl|
%   \c_keys_properties_path_tl|
%   \c_keys_utilities_path_tl|
% }
%   Paths for properties used by \pkg{l3keys} itself.
% \end{variable}
% 
% \begin{variable}{\c_keys_root_tl}
%   The root path for keys.
% \end{variable}
% 
% \begin{variable}{\l_keys_choice_code_tl}
%   The code to execute for each multiple choice when created \emph{en 
%   masse}.
% \end{variable}
% 
% \begin{variable}{
%   \l_keys_current_choice_tl|
%   \l_keys_current_choice_int|
% }
%   Information on multiple choices.
% \end{variable}
% 
% \begin{variable}{
%   \l_keys_current_key_full_tl|
%   \l_keys_current_key_name_tl|
% }
%   The current key name is stored both with and without a path.
% \end{variable}
% 
% \begin{variable}{\l_keys_current_module_tl}
%   Current module name used when creating csnames.
% \end{variable}
% 
% \begin{variable}{
%   \l_keys_current_path_tl|
%   \l_keys_default_path_tl|
%   \l_keys_choice_path_tl|
% }
%   Various key paths need to be stored.
% \end{variable}
% 
% \begin{variable}{\l_keys_current_value_toks}
%   The value given for the current key, stored as a token register.
% \end{variable}
% 
% \begin{variable}{\l_keys_no_value_bool}
%   A marker for ``no value'' as key input.
% \end{variable}
% 
% \begin{variable}{\l_keys_success_bool}
%   A marker used when trying keys without raising errors.
% \end{variable}
% 
% \begin{variable}{\l_keys_tmpa_tl}
%   A scratch variable.
% \end{variable}
% 
%\subsection{Internal functions}
%
% Notice that everything should be done using the keys system. Only
% \cs{keys_manage:n} and so on are intended for external use. In all
% cases, \m{key} is a fully-qualified key name. Functions created will
% be prefixed with \cs{c_keys_cs_prefix_tl}.
% 
% \begin{function}{\keys_bool_new:n}
%   \begin{syntax}
%     "\keys_bool_new:n" <key>
%   \end{syntax}
%   Creates a switch <key>, which will be set to \texttt{true}.
% \end{function}
% 
% \begin{function}{
%   \keys_bool_set:nN|
%   \keys_bool_set:nnn|
%  }
%   \begin{syntax}
%     "\keys_bool_set:nN" <function> <boolean>
%     "\keys_bool_set:nnn" <function> <prefix> <name>
%   \end{syntax}
%   Uses <function> (either \texttt{set} or \texttt{gset}) to use 
%   <value> in setting <boolean>. If a <name> is given, it is used to 
%   construct a csname for the switch including the current <module>. 
%   The <prefix> should be |l_| or |g_|.
% \end{function}
% 
% \begin{function}{
%   \keys_bool_set_inverse:n|
%   \keys_bool_set_inverse:nN|
%   \keys_bool_set_inverse:nnn|
%  }
%   \begin{syntax}
%     "\keys_bool_set:n" <switch>
%     "\keys_bool_set:nN" <function> <boolean>
%     "\keys_bool_set:nnn" <function> <prefix> <name>
%   \end{syntax}
%   Uses <function> (either \texttt{set} or \texttt{gset}) to use 
%   <value> in setting <boolean>. If a <name> is given, it is used to 
%   construct a csname for the switch including the current <module>. 
%   The <prefix> should be |l_| or |g_|. The logic of the setting is
%   reversed compared to the \cs{keys_bool_set:} functions.
% \end{function}
% 
% \begin{function}{\keys_clear_properties:n}
%   \begin{syntax}
%     "\keys_clear_properties:n" <key>
%   \end{syntax}
%   Clears the internal properties of <key>.
% \end{function}
% 
% \begin{function}{\keys_choice_create:n}
%   \begin{syntax}
%     "\keys_choice_create:n" <choice>
%   \end{syntax}
%   Creates <choice> as a sub-key of <key>.
% \end{function}
% 
% \begin{function}{\keys_choices_create:Nnn}
%   \begin{syntax}
%     "\keys_choices_create:Nnn" <expansion> <list> <code>
%   \end{syntax}
%   Takes a comma separated <list> and makes a <choice> as a sub-key of
%   the current <key> for each. The <choice> will execute <code>, which
%   is processed according to <expansion> (either \texttt{n} or 
%   \texttt{x}). The <code> has access to \cs{l_keys_current_choice_tl}
%   and \cs{l_keys_current_choice_int}, which indicate the choice given.
% \end{function}
% 
% \begin{function}{\keys_default_add:}
%   \begin{syntax}
%     "\keys_default_add:"
%   \end{syntax}
%   If no value was given for the current key, and a default value is
%   available, copies the default into \cs{l_keys_current_value_toks}.
% \end{function}
% 
% \begin{function}{\keys_msg_new:nn}
%   \begin{syntax}
%     "\keys_msg_new:nn" <name> <text>
%   \end{syntax}
%   Creates a new message with <name> which produces <text>.
% \end{function}
% 
% \begin{function}{
%   \keys_error:nn|
%   \keys_error:n|
% }
%   \begin{syntax}
%     "\keys_error:nn" <name> <arg>
%     "\keys_error:n" <name> 
%   \end{syntax}
%   Issues error <name>, with or without <argument>.
% \end{function}
% 
% \begin{function}{
%   \keys_find_code_full:|
%   \keys_find_code_name:
% }
%   \begin{syntax}
%     "\keys_find_code_full:"
%     "\keys_find_code_name:"
%   \end{syntax}
%   Find code to execute for the current fully qualified key 
%   (\texttt{full}) or key name only (\texttt{name}).
% \end{function}
% 
% \begin{function}{\keys_if_cmd_exist:n / (TF)}
%   \begin{syntax}
%     "\keys_if_cmd_exist:nTF" <key> <true code> 
%     ~~~~<false code>
%   \end{syntax}
%   Checks for the existence of a |._cmd:w| function for <key>.
% \end{function}
% 
% \begin{function}{\keys_if_exist:n / (TF)}
%   \begin{syntax}
%     "\keys_if_exist:nTF" <key> <true code> 
%     ~~~~<false code>
%   \end{syntax}
%   Checks for the existence of a <key>.
% \end{function}
% 
% \begin{function}{\keys_if_value:n / (TF)}
%   \begin{syntax}
%     "\keys_if_value:nTF" <property> <true code> <false code>
%   \end{syntax}
%   Checks for the <property> switch of a <key> (typically 
%   \texttt{required} or \texttt{forbidden}).
% \end{function}
% 
% \begin{function}{
%   \keys_no_value_elt:n|
%   \keys_value_elt:nn
% }
%   \begin{syntax}
%     "\keys_no_value_elt:n" <elt>
%     "\keys_value_elt:nn" <elt> <value>
%   \end{syntax}
%   Functions used by \pkg{l3keyval} for each <elt> of the <keyval 
%   list> being processed.
% \end{function}
% 
% \begin{function}{\keys_parse:n}
%   \begin{syntax}
%     "\keys_parse:n" <keyval list>
%   \end{syntax}
%   Functions which actually parses <keyval list>.
% \end{function}
% 
% \begin{function}{\keys_parse_list:n}
%   \begin{syntax}
%     "\keys_parse_list:n" <keyval list>
%   \end{syntax}
%   Set up macro for \cs{keys_parse:n}.
% \end{function}
% 
% \begin{function}{
%   \keys_path_add:N|
%   \keys_path_add:w
%  }
%   \begin{syntax}
%     "\keys_path_add:N" <tl>
%     "\keys_path_add:w" <tlist> "\q_stop"
%   \end{syntax}
%   Adds a full path to key name stored in <tl> or given as <tlist>.
% \end{function}
% 
% \begin{function}{\keys_process_elt:nn}
%   \begin{syntax}
%     "\keys_process_elt:nn" <elt> <value>
%   \end{syntax}
%   Lead-off processor for converting <elt> into a fully-qualified 
%   <key> and checking validity of <value>.
% \end{function}
% 
% \begin{function}{
%   \keys_separate_path:|
%   \keys_separate_path:w
%  }
%   \begin{syntax}
%     "\keys_separate_path:" 
%     "\keys_separate_path:w" / <path 1> / <path 2> "\q_stop"
%   \end{syntax}
%   Separates key into <path> and <key name>.
% \end{function}
% 
% \begin{function}{
%   \keys_set:NN|
%   \keys_set:Nnn|
%  }
%   \begin{syntax}
%     "\keys_set:NN" <function> <variable>
%     "\keys_set:Nnn" <function> <prefix> <name>
%   \end{syntax}
%   Uses <function> to store <value> in <variable>. If a <name> is
%   given, it is used to construct a csname for the variable including
%   the current <module>. The <prefix> should be |l_| or |g_|.
% \end{function}
% 
% \begin{function}{
%   \keys_set_cmd:nn|
%   \keys_set_cmd:nx|
%   \keys_set_cmd:nNn|
%   \keys_set_cmd:nNx|
%  }
%   \begin{syntax}
%     "\keys_set_cmd:nn" <key> <code>
%     "\keys_set_cmd:nNn" <key> <number> <code>
%   \end{syntax}
%   Creates a |._cmd:w| function for <key>, with definition <code>. The
%   \texttt{N} variant can include <number> parameters.
% \end{function}
% 
% \begin{function}{
%   \keys_store:nn|
%   \keys_store:nx
%  }
%   \begin{syntax}
%     "\keys_store:nn" <key> <data>
%   \end{syntax}
%   Stores <data> in <key> function.
% \end{function}
% 
% \begin{function}{\keys_toks_set:Nn}
%   \begin{syntax}
%     "\keys_toks_set:Nn" <toks> <key>
%   \end{syntax}
%   Sets <toks> equal to the content of <key>
% \end{function}
% 
% \begin{function}{\keys_try:}
%   \begin{syntax}
%     "\keys_try:"
%   \end{syntax}
%   Attempt to execute a key, with no error if the key is unknown.
% \end{function}
% 
% \begin{function}{\keys_undefine:n}
%   \begin{syntax}
%     "\keys_undefine:n" <key>
%   \end{syntax}
%   Delete definition of <key>.
% \end{function}
% 
% \begin{function}{\keys_use:n}
%   \begin{syntax}
%     "\keys_use:n" <key>
%   \end{syntax}
%   Use definition of <key>.
% \end{function}
% 
% \begin{function}{
%   \keys_use_cmd:n|
%   \keys_use_cmd:nn
% }
%   \begin{syntax}
%     "\keys_use_cmd:n" <key>
%     "\keys_use_cmd:nn" <key> <arg>
%   \end{syntax}
%   Uses the |._cmd:w| function for <key>, passing <arg> if needed.
% \end{function}
% 
%\subsubsection{Internal properties}
%
% The internal key properties should not be accessed directly.
%
% \begin{function}{._cmd:w}
%  \begin{syntax}
%  \end{syntax}
%  The function which is executed for a key with \texttt{.code}.
% \end{function}
%
% \begin{function}{._default_tl}
%  \begin{syntax}
%  \end{syntax}
%  Holds the default value for a key.
% \end{function}
% 
% \begin{function}{._forbidden_bool}
%  \begin{syntax}
%  \end{syntax}
%  Indicates that a value cannot be given for <key>.
% \end{function}
% 
% \begin{function}{._num_args_tl}
%  \begin{syntax}
%  \end{syntax}
%  For functions defined with \texttt{.code:Nn} and \texttt{.code:Nx},
%  contains the number of arguments the associated |._cmd:w| function
%  takes.
% \end{function}
% 
% \begin{function}{._required_bool}
%  \begin{syntax}
%  \end{syntax}
%  Indicates that <key> must have a value provided.
% \end{function}
% 
%\subsection{Implementation}
%
% The usual preliminaries.  The key--value parsing itself is handled by
% \pkg{l3keyval}, which does the very low-level stuff so there is no
% need to worry here.
%    \begin{macrocode}
%<*package>
\ProvidesExplPackage
  {\filename}{\filedate}{\fileversion}{\filedescription}
\RequirePackage{l3keyval,l3msg,l3clist,l3skip}
%    \end{macrocode}
%
%\subsubsection{Variables and contrasts}
%
%\begin{macro}{\c_keys_0_empty_tl}
%\begin{macro}{\c_keys_1_empty_tl}
%\begin{macro}{\c_keys_2_empty_tl}
%\begin{macro}{\c_keys_3_empty_tl}
%\begin{macro}{\c_keys_4_empty_tl}
%\begin{macro}{\c_keys_5_empty_tl}
%\begin{macro}{\c_keys_6_empty_tl}
%\begin{macro}{\c_keys_7_empty_tl}
%\begin{macro}{\c_keys_8_empty_tl}
%\begin{macro}{\c_keys_9_empty_tl}
% A set of empty arguments.
%    \begin{macrocode}
\tl_new:cn { c_keys_0_empty_tl } {}
\tl_new:cn { c_keys_1_empty_tl } { {} }
\tl_new:cn { c_keys_2_empty_tl } { {} {} }
\tl_new:cn { c_keys_3_empty_tl } { {} {} {} }
\tl_new:cn { c_keys_4_empty_tl } { {} {} {} {} }
\tl_new:cn { c_keys_5_empty_tl } { {} {} {} {} {} }
\tl_new:cn { c_keys_6_empty_tl } { {} {} {} {} {} {} }
\tl_new:cn { c_keys_7_empty_tl } { {} {} {} {} {} {} {} }
\tl_new:cn { c_keys_8_empty_tl } { {} {} {} {} {} {} {} {} }
\tl_new:cn { c_keys_9_empty_tl } { {} {} {} {} {} {} {} {} {} }
%    \end{macrocode}
%\end{macro}
%\end{macro}
%\end{macro}
%\end{macro}
%\end{macro}
%\end{macro}
%\end{macro}
%\end{macro}
%\end{macro}
%\end{macro}
%
%\begin{macro}{\c_keys_cs_prefix_tl}
% First, the small number of constants needed are created. A prefix is
% used to keep all of the actual key macros in one place.
%    \begin{macrocode}
\tl_new:Nn \c_keys_cs_prefix_tl { keys-root }
%    \end{macrocode}
%\end{macro}
%
%\begin{macro}{\c_keys_errors_path_tl}
%\begin{macro}{\c_keys_properties_path_tl}
%\begin{macro}{\c_keys_utilities_path_tl}
% The locations of all of the keys used by \pkg{keys3} itself.
%    \begin{macrocode}
\tl_new:Nn \c_keys_errors_path_tl     { /keys/errors }
\tl_new:Nn \c_keys_properties_path_tl { /keys/properties }
\tl_new:Nn \c_keys_utilities_path_tl  { /keys }
%    \end{macrocode}
%\end{macro}
%\end{macro}
%\end{macro}
%
%\begin{macro}{\c_keys_root_tl}
% The key root should have a clear name; like all of the key macros,
% this does not include the prefix.
%    \begin{macrocode}
\tl_new:Nn \c_keys_root_tl { / }
%    \end{macrocode}
%\end{macro}
%
%\begin{macro}{\l_keys_choice_code_tl}
% When making choices, the code for each key has to be stored.
%    \begin{macrocode}
\tl_new:N \l_keys_choice_code_tl
%    \end{macrocode}
%\end{macro}
%
%\begin{macro}{\l_keys_current_choice_tl}
%\begin{macro}{\l_keys_current_choice_int}
% Multiple choices need some storage.
%    \begin{macrocode}
\tl_new:N \l_keys_current_choice_tl
\int_new:N \l_keys_current_choice_int
%    \end{macrocode}
%\end{macro}
%\end{macro}
%
%\begin{macro}{\l_keys_current_key_full_tl}
%\begin{macro}{\l_keys_current_key_name_tl}
% The current key name and the fully-qualified key are stored.
%    \begin{macrocode}
\tl_new:N \l_keys_current_key_full_tl
\tl_new:N \l_keys_current_default_tl
%    \end{macrocode}
%\end{macro}
%\end{macro}
%
%\begin{macro}{\l_keys_current_module_tl}
% The module name of the current module is stored here.
%    \begin{macrocode}
\tl_new:N \l_keys_current_module_tl
%    \end{macrocode}
%\end{macro}
%
%\begin{macro}{\l_keys_current_path_tl}
%\begin{macro}{\l_keys_default_path_tl}
%\begin{macro}{\l_keys_choice_path_tl}
% The current and default paths can be stored as tls. The default path
% is then initialised as the key root.
%    \begin{macrocode}
\tl_new:N \l_keys_current_path_tl
\tl_new:N \l_keys_default_path_tl
\tl_set_eq:NN \l_keys_default_path_tl \c_keys_root_tl
\tl_new:N \l_keys_choice_path_tl
%    \end{macrocode}
%\end{macro}
%\end{macro}
%\end{macro}
%
%\begin{macro}{\l_keys_current_value_toks}
% The current value is stored in a token register.
%    \begin{macrocode}
\toks_new:N \l_keys_current_value_toks
%    \end{macrocode}
%\end{macro}
%
%\begin{macro}{\l_keys_no_value_bool}
% To indicate that no value has been given.
%    \begin{macrocode}
\bool_new:N \l_keys_no_value_bool
%    \end{macrocode}
%\end{macro}
%
%\begin{macro}{\l_keys_success_bool}
% A switch for trying keys.
%    \begin{macrocode}
\bool_new:N \l_keys_success_bool
%    \end{macrocode}
%\end{macro}
%
%\begin{macro}{\l_keys_tmpa_tl}
% A scratch area.
%    \begin{macrocode}
\tl_new:N \l_keys_tmpa_tl
%    \end{macrocode}
%\end{macro}
%
%\subsubsection{Functions}
%
%\begin{macro}{\keys_manage:n}
%\begin{macro}{\keys_manage_quick:n}
%\begin{macro}{\keys_manage_internal:n}
%\begin{macro}[aux]{\keys_manage_aux:nn}
%\begin{macro}[aux]{\keys_manage_aux:Vn}
% The main key management macros both call the auxiliary function after
% setting up the parser.  The expansion trick means a literal path is
% sent to the later function, and so the default path can be redefined.
%    \begin{macrocode}
\cs_new:Nn \keys_manage:n {
  \cs_set_eq:NN \keys_parse:n \KV_parse_space_removal_sanitize:n
  \tl_clear:N \l_keys_current_module_tl
  \keys_manage_internal:n {#1}
}
\cs_new:Nn \keys_manage_quick:n {
  \cs_set_eq:NN \keys_parse:n \KV_parse_no_space_removal_no_sanitize:n
  \tl_clear:N \l_keys_current_module_tl
  \keys_manage_internal:n {#1}
}
\cs_new:Nn \keys_manage_internal:n {
  \keys_manage_aux:Vn \l_keys_default_path_tl {#1}
}
\cs_new:Nn \keys_manage_aux:nn {
  \tl_set_eq:NN \l_keys_default_path_tl \c_keys_root_tl
  \keys_parse_list:n {#2}
  \tl_set:Nn \l_keys_default_path_tl {#1}
}
\cs_generate_variant:Nn \keys_manage_aux:nn { Vn }
%    \end{macrocode}
%\end{macro}
%\end{macro}
%\end{macro}
%\end{macro}
%\end{macro}
%\begin{macro}{\keys_manage:nn}
%\begin{macro}{\keys_manage_quick:nn}
%\begin{macro}[aux]{\keys_manage_aux:nnn}
%\begin{macro}[aux]{\keys_manage_aux:Vnn}
% This version uses the same tricks but includes the path as a second
% argument.  When setting lots of keys separately, this is a little
% faster than the key-based method.
%    \begin{macrocode}
\cs_new:Nn \keys_manage:nn {
  \cs_set_eq:NN \keys_parse:n \KV_parse_space_removal_sanitize:n
  \keys_manage_aux:Vnn \l_keys_default_path_tl {#1} {#2}
}
\cs_new:Nn \keys_manage_quick:nn {
  \cs_set_eq:NN \keys_parse:n \KV_parse_space_no_removal_no_sanitize:n
  \keys_manage_aux:Vnn \l_keys_default_path_tl {#1} {#2}
}
\cs_new:Nn \keys_manage_aux:nnn {
  \tl_set:Nn \l_keys_default_path_tl {#2}
  \tl_clear:N \l_keys_current_module_tl
  \keys_parse_list:n {#3}
  \tl_set:Nn \l_keys_default_path_tl {#1}
}
\cs_generate_variant:Nn \keys_manage_aux:nnn { Vnn }
%    \end{macrocode}
%\end{macro}
%\end{macro}
%\end{macro}
%\end{macro}
%
%\subsubsection{Internal functions}
%
%\begin{macro}{\keys_bool_new:n}
% To create a new switch, which will be true (as the existence of the 
% switch is the flag here).
%    \begin{macrocode}
\cs_new_nopar:Nn \keys_bool_new:n {
  \bool_new:c { \c_keys_cs_prefix_tl #1 }
  \bool_set_true:c { \c_keys_cs_prefix_tl #1 }
}
%    \end{macrocode}
%\end{macro}
%
%\begin{macro}{\keys_bool_set:nN}
%\begin{macro}{\keys_bool_set:nnn}
% Boolean keys are created by using the fact that only \texttt{true} and
% \texttt{false} give the right result when looking for a setting 
% function. A default is also set, so that the key name alone can be 
% given.
%    \begin{macrocode}
\cs_new_nopar:Nn \keys_bool_set:nN {
  \keys_parse_list:n {
    \c_keys_properties_path_tl /.code:n = {
      \cs_if_exist:cTF { bool_ #1 _ ##1 :N } {
        \use:c { bool_ #1 _ ##1 :N } #2
      }{
        \keys_error:nn { Boolean~expected } {##1}
      }
    },
    \l_keys_current_path_tl /.default:n = true
  }
}
\cs_new_nopar:Nn \keys_bool_set:nnn {
  \keys_parse_list:n {
    \c_keys_properties_path_tl /.code:x = {
      \exp_not:N \cs_if_exist:cTF { bool_ #1 _ ##1 :c } {
        \exp_not:N \use:c { bool_ #1 _ ##1 :c } 
          { #2 \l_keys_current_module_tl #3 _bool }
      }{
        \keys_error:nn { Boolean~expected } {##1}
      }
    },
    \l_keys_current_path_tl /.default:n = true
  }
}
%    \end{macrocode}
%\end{macro}
%\end{macro}
%
%\begin{macro}{\keys_bool_set_inverse:n}
%\begin{macro}{\keys_bool_set_inverse:nN}
%\begin{macro}{\keys_bool_set_inverse:nnn}
% To set keys with reversed logic, the basics can be done in the same 
% way as for the standard switches.  The only thing needed is a 
% reversal of \texttt{true}/\texttt{false}.
%    \begin{macrocode}
\cs_new_nopar:Nn \keys_bool_set_inverse:n {
  \tlist_if_eq:nnTF {#1} { true } { false } { true }
}
\cs_new_nopar:Nn \keys_bool_set_inverse:nN {
  \keys_parse_list:n {
    \c_keys_properties_path_tl /.code:n = {
      \cs_if_exist:cTF { bool_ #1 _ ##1 :N } {
        \use:c { bool_ #1 _ \keys_bool_set_inverse:n {##1} :N } #2
      }{
        \keys_error:nn { Boolean~expected } {##1}
      }
    },
    \l_keys_current_path_tl /.default:n = true
  }
}
\cs_new_nopar:Nn \keys_bool_set_inverse:nnn {
  \keys_parse_list:n {
    \c_keys_properties_path_tl /.code:x = {
      \exp_not:N \cs_if_exist:cTF { bool_ #1 _ ##1 :c } {
         \exp_not:N \use:c { 
           bool_ #1 _ \exp_not:N \keys_bool_set_inverse:n {##1} :c 
         } { #2 \l_keys_current_module_tl #3 _bool }
      }{
        \keys_error:nn { Boolean~expected } {##1}
      }
    },
    \l_keys_current_path_tl /.default:n = true
  }
}
%    \end{macrocode}
%\end{macro}
%\end{macro}
%\end{macro}
%
%\begin{macro}{\keys_choices_create:Nnn}
%\begin{macro}[aux]{\keys_choices_create_aux:n}
%\begin{macro}[aux]{\keys_choices_create_aux:x}
%\begin{macro}{\keys_choice_create:n}
% When making multiple choices, the code for each choice is the same.
% Only the path and counter need to be altered. 
%    \begin{macrocode}
\cs_new:Nn \keys_choices_create:Nnn {
  \tl_set_eq:NN \l_keys_choice_path_tl \l_keys_current_path_tl
  \int_zero:N \l_keys_current_choice_int
  \use:c { keys_choices_create_aux: #1 } {#3}
  \clist_map_function:nN {#2} \keys_choice_create:n
  \keys_manage_internal:n { \l_keys_choice_path_tl  /.expects_choice: }
}
\cs_new:Nn \keys_choices_create_aux:n {
  \tl_set:Nn \l_keys_choice_code_tl { \exp_not:n {#1} }
}
\cs_new:Nn \keys_choices_create_aux:x {
  \tl_set:Nn \l_keys_choice_code_tl {#1}
}
\cs_new_nopar:Nn \keys_choice_create:n {
  \int_incr:N \l_keys_current_choice_int
  \keys_parse_list:n {
    \l_keys_choice_path_tl / #1 /.code:x = {
      \exp_not:n { \int_set:Nn \l_keys_current_choice_int }
        { \int_use:N \l_keys_current_choice_int }
      \exp_not:n { \tl_set:Nn \l_keys_current_choice_tl } {#1}
      \l_keys_choice_code_tl
    }
  }
}
%    \end{macrocode}
%\end{macro}
%\end{macro}
%\end{macro}
%\end{macro}
%
%\begin{macro}{\keys_clear_properties:n}
% To avoid problem on redefinition, all properties are removed.
%    \begin{macrocode}
\cs_new_nopar:Nn \keys_clear_properties:n {
  \keys_undefine:n { #1 /._cmd:w }      
  \keys_undefine:n { #1 /._default_tl }  
  \keys_undefine:n { #1 /._forbidden_bool } 
  \keys_undefine:n { #1 /._num_args_tl }
  \keys_undefine:n { #1 /._required_bool }
}
%    \end{macrocode}
%\end{macro}
%
%\begin{macro}{\keys_default_add:}
% Copies the default value to the current one if needed.
%    \begin{macrocode}
\cs_new_nopar:Nn \keys_default_add: {
  \bool_if:NT \l_keys_no_value_bool {
    \keys_if_exist:nT 
      { \l_keys_current_key_full_tl /._default_tl } {
      \keys_toks_set:Nn \l_keys_current_value_toks
        { \l_keys_current_key_full_tl /._default_tl }
      \bool_set_false:N \l_keys_no_value_bool
    }
  }
}
%    \end{macrocode}
%\end{macro}
%
%\begin{macro}{\keys_find_code_full:}
%\begin{macro}{\keys_find_code_name:}
% Two functions to find something to process the key value given. First,
% a search is made for either a command property for the key, or a 
% function for the key itself. If that fails, the generic handlers are 
% used after separating out the key name and key path.
%    \begin{macrocode}
\cs_new_nopar:Nn \keys_find_code_full: {
  \keys_if_cmd_exist:nTF { \l_keys_current_key_full_tl } {
    \keys_use_cmd:n { \l_keys_current_key_full_tl }
  }{
    \keys_if_exist:nTF { \l_keys_current_key_full_tl } {
      \bool_if:NTF \l_keys_no_value_bool {
        \keys_use:n { \l_keys_current_key_full_tl }
      }{
        \keys_store:nx { \l_keys_current_key_full_tl }
          { \toks_use:N \l_keys_current_value_toks }
      }
    }{
      \keys_find_code_name:
    }
  }
}
\cs_new_nopar:Nn \keys_find_code_name: {
  \keys_separate_path:
  \keys_if_cmd_exist:nTF
    { \c_keys_properties_path_tl / \l_keys_current_key_name_tl } {
    \keys_use_cmd:n  
      {\c_keys_properties_path_tl / \l_keys_current_key_name_tl}
  }{
    \keys_if_cmd_exist:nTF 
      { \l_keys_current_path_tl / unknown } {
      \keys_use_cmd:n { \l_keys_current_path_tl / unknown }
    }{
      \keys_error:n { unknown~key }
    }
  }
}
%    \end{macrocode}
%\end{macro}
%\end{macro}
%
%\begin{macro}[TF]{\keys_if_cmd_exist:n}
% A dedicated check for the |._cmd:w| property key.
%    \begin{macrocode}
\prg_new_conditional:Nnn \keys_if_cmd_exist:n {TF,T,F} {
 \if_cs_exist:w \c_keys_cs_prefix_tl #1 /._cmd:w \cs_end:
   \prg_return_true: 
  \else: 
    \prg_return_false: 
  \fi:
}
%    \end{macrocode}
%\end{macro}
%
%\begin{macro}[TF]{\keys_if_exist:n}
% Check if a key exists without adding to the hash table.
%    \begin{macrocode}
\prg_new_conditional:Nnn \keys_if_exist:n {TF,T,F} {
 \if_cs_exist:w \c_keys_cs_prefix_tl #1 \cs_end:
   \prg_return_true: 
  \else: 
    \prg_return_false: 
  \fi:
}
%    \end{macrocode}
%\end{macro}
%
%\begin{macro}[TF]{\keys_if_value:n}
% Used for required and forbidden values.
%    \begin{macrocode}
\prg_new_conditional:Nnn \keys_if_value:n {TF,T,F} {
 \if_cs_exist:w 
   \c_keys_cs_prefix_tl \l_keys_current_key_full_tl /._ #1 _bool 
 \cs_end:
   \prg_return_true: 
  \else: 
    \prg_return_false: 
  \fi:
}
%    \end{macrocode}
%\end{macro}
%
%\begin{macro}{\keys_no_value_elt:n}
%\begin{macro}{\keys_value_elt:nn}
% The two functions passed to \pkg{l3keyval} to actually act on each
% key found.
%    \begin{macrocode}
\cs_new:Nn \keys_no_value_elt:n {
  \bool_set_true:N \l_keys_no_value_bool
  \keys_process_elt:nn {#1} { }
}
\cs_new:Nn \keys_value_elt:nn {
  \bool_set_false:N \l_keys_no_value_bool
  \keys_process_elt:nn {#1} {#2}
}
%    \end{macrocode}
%\end{macro}
%\end{macro}
%
%\begin{macro}{\keys_parse:n}
% The macro used to actually process the key--value input is taken from
% \pkg{l3keyval}.  There are two possible options, and so at this
% stage the macro is simply reserved.
%    \begin{macrocode}
\cs_new:Nn \keys_parse:n { \ERROR }
%    \end{macrocode}
%\end{macro}
%
%\begin{macro}{\keys_parse_list:n}
% All of the management macros call this common parser. First, the key
% processing macros are defined, then the appropriate parser is called.
%    \begin{macrocode}
\cs_new:Nn \keys_parse_list:n {
  \cs_set_eq:NN \KV_key_value_elt:nn \keys_value_elt:nn
  \cs_set_eq:NN \KV_key_no_value_elt:n \keys_no_value_elt:n
  \keys_parse:n {#1}
}
%    \end{macrocode}
%\end{macro}
%
%\begin{macro}{\keys_path_add:N}
%\begin{macro}{\keys_path_add:w}
%\begin{macro}[aux]{\keys_path_add_aux:w}
% The code to check for a path looks for a |/| at the start of the key.
%    \begin{macrocode}
\cs_new_nopar:Nn \keys_path_add:N {
  \exp_after:wN \keys_path_add:w #1 \q_stop
}
\cs_new_nopar:Npn \keys_path_add:w {
  \exp_after:wN \peek_meaning:NTF \c_keys_root_tl {
    \use_none_delimit_by_q_stop:w
  }{
    \keys_path_add_aux:w
  }
}
\cs_new_nopar:Npn \keys_path_add_aux:w #1 \q_stop {
  \tl_set:Nx \l_keys_current_key_full_tl {\l_keys_default_path_tl #1}
}
%    \end{macrocode}
%\end{macro}
%\end{macro}
%\end{macro}
%
%\begin{macro}{\keys_process_elt:nn}
%\begin{macro}[aux]{\keys_process_elt_aux:}
% The key processor starts by storing the given key name and value, and
% adding a path to the former if necessary. There is then potentially a
% need to fill in a default value before checking for required or 
% forbidden values.
%    \begin{macrocode}
\cs_new:Nn \keys_process_elt:nn {
  \tl_set:Nx \l_keys_current_key_full_tl {#1}
  \toks_set:Nn \l_keys_current_value_toks {#2}
  \keys_path_add:N \l_keys_current_key_full_tl
  \keys_default_add:
  \keys_if_value:nTF { required } {
    \bool_if:NTF \l_keys_no_value_bool {
      \keys_error:n { value~required }
    }{
      \keys_process_elt_aux:
    }
  }{
    \keys_process_elt_aux:
  }
}
\cs_new:Nn \keys_process_elt_aux: {
  \keys_if_value:nTF { forbidden } {
    \bool_if:NTF \l_keys_no_value_bool {
      \keys_find_code_full:
    }{
      \keys_error:nn { value~forbidden } 
        { \toks_use:N \l_keys_current_value_toks }
    }
  }{
    \keys_find_code_full:
  }
}
%    \end{macrocode}
%\end{macro}
%\end{macro}
%
%\begin{macro}{\keys_separate_path:}
%\begin{macro}{\keys_separate_path:w}
% A simple piece of recursion to find the key name and path.
%    \begin{macrocode}
\cs_new_nopar:Nn \keys_separate_path: {
  \tl_clear:N \l_keys_current_path_tl
  \exp_after:wN \keys_separate_path:w \l_keys_current_key_full_tl 
    / \q_stop
}
\cs_new_nopar:Npn \keys_separate_path:w / #1 / #2 \q_stop {
  \tlist_if_empty:nTF {#2} {
    \tl_set:Nn \l_keys_current_key_name_tl {#1}
  }{
     \tl_put_right:Nn \l_keys_current_path_tl { / #1 }
     \keys_separate_path:w /#2 \q_stop
  }
}
%    \end{macrocode}
%\end{macro}
%\end{macro}
%
%\begin{macro}{\keys_set:NN}
%\begin{macro}{\keys_set:Nnn}
% Storage of data in outside of keys.
%    \begin{macrocode}
\cs_new_nopar:Nn \keys_set:NN {
  \keys_parse_list:n {
    \c_keys_properties_path_tl /.code:n = {
      #1 #2 {##1}
    }
  }
}
\cs_new_nopar:Nn \keys_set:Nnn {
  \keys_parse_list:n {
    \c_keys_properties_path_tl /.code:x = {
      \exp_not:N #1 { #2 \l_keys_current_module_tl #3 } {##1}
    }
  }
}
%    \end{macrocode}
%\end{macro}
%\end{macro}
%
%\begin{macro}{\keys_set_cmd:nn}
%\begin{macro}{\keys_set_cmd:nx}
%\begin{macro}{\keys_set_cmd:nNn}
%\begin{macro}{\keys_set_cmd:nNx}
%\begin{macro}{._cmd:w}
%\begin{macro}{._num_args_tl}
% Creation of the key |._cmd:w| macros happens here. For the 
% multiple-argument versions, the number of arguments is stored for
% use later on.
%    \begin{macrocode}
\cs_new:Nn \keys_set_cmd:nn {
  \keys_clear_properties:n {#1}
  \cs_set:cNn { \c_keys_cs_prefix_tl #1 /._cmd:w } 1 {#2}
}
\cs_new:Nn \keys_set_cmd:nx {
  \keys_clear_properties:n {#1}
  \cs_set:cNx { \c_keys_cs_prefix_tl #1 /._cmd:w } 1 {#2}
}
\cs_new:Nn \keys_set_cmd:nNn {
  \keys_clear_properties:n {#1}
  \cs_set:cNn { \c_keys_cs_prefix_tl #1 /._cmd:w } #2 {#3}
  \keys_store:nn { #1 /._num_args_tl} {#2}
}
\cs_new:Nn \keys_set_cmd:nNx {
  \keys_clear_properties:n {#1}
  \cs_set:cNx { \c_keys_cs_prefix_tl #1 /._cmd:w } #2 {#3}
  \keys_store:nn { #1 /._num_args_tl} {#2}
}
%    \end{macrocode}
%\end{macro}
%\end{macro}
%\end{macro}
%\end{macro}
%\end{macro}
%\end{macro}
%
%\begin{macro}{\keys_store:nn}
%\begin{macro}{\keys_store:nx}
% Direct storage of data in keys.
%    \begin{macrocode}
\cs_new:Nn \keys_store:nn {
  \tl_set:cn { \c_keys_cs_prefix_tl #1 } {#2}
}
\cs_new:Nn \keys_store:nx {
  \tl_set:cx { \c_keys_cs_prefix_tl #1 } {#2}
}
%    \end{macrocode}
%\end{macro}
%\end{macro}
%
%\begin{macro}{\keys_toks_set:Nn}
% Sets a toks to the content of a key.
%    \begin{macrocode}
\cs_new_nopar:Nn \keys_toks_set:Nn {
  \exp_args:NNv \toks_set:Nn #1 { \c_keys_cs_prefix_tl #2 }
}
%    \end{macrocode}
%\end{macro}
%
%\begin{macro}{\keys_try:}
% A function to look for the |._cmd:w| property of a key, and executes
% it if found.
%    \begin{macrocode}
\cs_new_nopar:Nn \keys_try: {
  \bool_set_false:N \l_keys_success_bool
  \keys_if_cmd_exist:nT { \l_keys_current_path_tl } {
    \bool_set_true:N \l_keys_success_bool
    \bool_if:NT \l_keys_no_value_bool {
      \keys_if_exist:nT
        { \l_keys_current_path_tl /._default_tl } {
        \keys_toks_set:Nn \l_keys_current_value_toks
          { \l_keys_current_path_tl /._default_tl }
      }
    }
    \cs_set_eq:NN \l_keys_current_key_full_tl \l_keys_current_path_tl 
    \keys_use_cmd:n { \l_keys_current_path_tl }
  }
}
%    \end{macrocode}
%\end{macro}
%
%\begin{macro}{\keys_undefine:n}
% To remove a key.
%    \begin{macrocode}
\cs_new_nopar:Nn \keys_undefine:n {
  \cs_set_eq:cN { \c_keys_cs_prefix_tl #1 } \c_undefined
}
%    \end{macrocode}
%\end{macro}
%
%\begin{macro}{\keys_use:n}
% Use whatever is stored in a key.
%    \begin{macrocode}
\cs_new_nopar:Nn \keys_use:n {
  \use:c { \c_keys_cs_prefix_tl #1 }
}
%    \end{macrocode}
%\end{macro}
%
%\begin{macro}{\keys_use_cmd:n}
%\begin{macro}{\keys_use_cmd:nn}
%\begin{macro}[aux]{\keys_use_cmd_aux:nn}
%\begin{macro}[aux]{\keys_use_cmd_aux:w}
% Some care is needed when using command keys. For commands with 
% multiple arguments a check is made in case none were given, and if
% so a series of empty values is given instead.
%    \begin{macrocode}
\cs_new_nopar:Nn \keys_use_cmd:n {
  \exp_args:Nno \keys_use_cmd:nn {#1} 
    { \toks_use:N \l_keys_current_value_toks }
}
\cs_new_nopar:Nn \keys_use_cmd:nn {
  \keys_if_exist:nTF { #1 /._num_args_tl } {
     \keys_use_cmd_aux:nn {#1} {#2}
  }{
    \keys_use:n {#1/._cmd:w} {#2}
  }
}
\cs_new_nopar:Nn \keys_use_cmd_aux:nn {
  \tlist_if_empty:nTF {#2} {
    \cs_set:NNn \keys_use_cmd_aux:w 0 {
      \keys_use:n { #1 /._cmd:w }
    }
    \exp_after:wN \exp_after:wN \exp_after:wN \keys_use_cmd_aux:w 
      \cs:w
        c_keys_ \keys_use:n { #1 /._num_args_tl } _empty_tl
      \cs_end:
  }{
    \keys_use:n { #1 /._cmd:w } #2
  }
}
%    \end{macrocode}
%\end{macro}
%\end{macro}
%\end{macro}
%\end{macro}
%
%\subsubsection{Error handling code}
%
% The \LaTeX3 approach is to have named errors called separately and 
% defined separately. To make life a little easier here, some custom
% functions are used to keep repetition down.
% 
%\begin{macro}{\keys_msg_new:nn}
% To create new error messages, a utility function is created.
%    \begin{macrocode}
\cs_new_nopar:Npn \keys_msg_new:nn #1 {
  \msg_new:nnn { keys } {#1}
}
%    \end{macrocode}
%\end{macro}
%
%\begin{macro}{\keys_error:n}
%\begin{macro}{\keys_error:nn}
% Utilities for using errors, which always have 
% \cs{l_keys_current_key_full_tl} as the first argument.
%    \begin{macrocode}
\cs_new_protected_nopar:Nn \keys_error:nn {
  \msg_error:nnnn { keys } {#1} { \l_keys_current_key_full_tl } {#2}
}
\cs_new_protected_nopar:Nn \keys_error:n {
  \msg_error:nnnn { keys } {#1} { \l_keys_current_key_full_tl }
}
%    \end{macrocode}
%\end{macro}
%\end{macro}
%
%    \begin{macrocode}
\keys_msg_new:nn 
  { unknown~key } 
  {The key `#1' is unknown and is being ignored.}
\keys_msg_new:nn 
  { value~required }
  {The key `#1' requires a value\\%
  and is being ignored.}
\keys_msg_new:nn 
  { value~forbidden }
  {The key `#1' cannot taken a value:\\%
   the given input `#2' is being ignored.}
\keys_msg_new:nn 
  { Boolean~expected } 
  {Key `#1' takes the Boolean values\\%
  `true' and `false'~only.\\%
   The given value `#2' is being ignored.}
\keys_msg_new:nn 
  { not_boolean }
  {Key `#1' is not a Boolean key: you cannot create a complement.}
\keys_msg_new:nn 
  { unknown~choice }
  {Choice `#2' unknown for key `#1':\\%
   the key is being ignored.}
%    \end{macrocode}
%
%\subsubsection{Property definitions}
%
%\begin{macro}{.code:n}
%\begin{macro}{.code:Nn}
% The \texttt{.code:n} and \texttt{.code:Nn} properties have to be 
% defined directly.
%    \begin{macrocode}
\keys_set_cmd:nn { \c_keys_properties_path_tl /.code:n } {
  \keys_set_cmd:nn { \l_keys_current_path_tl } {#1}
}
\keys_set_cmd:nNn { \c_keys_properties_path_tl /.code:Nn } 2 {
  \keys_set_cmd:nNn { \l_keys_current_path_tl } #1 {#2}
}
%    \end{macrocode}
%\end{macro}
%\end{macro}
%
% The remaining definitions can all be carried out using the package
% itself. As category codes and spaces are not an issue here, the
% |_quick| version of \cs{keys_manage} is used.
% 
% First, an error is created for unknown keys: this is done early to
% catch any internal errors.
%    \begin{macrocode}
\keys_manage_quick:n {
  \c_keys_errors_path_tl /unknown/.code:n = {
    \keys_error:n { unknown~key }
  }
}
%    \end{macrocode}
%    
%\begin{macro}{.code:x}
%\begin{macro}{.code:Nx}
% Fully-expanded versions of the basic \texttt{.code} properties.
%    \begin{macrocode}
\keys_manage_quick:n {
  \c_keys_properties_path_tl /.code:x/.code:n = {
    \keys_set_cmd:nx { \l_keys_current_path_tl } {#1}
  },
  \c_keys_properties_path_tl /.code:Nx/.code:Nn = 2 {
    \keys_set_cmd:nNx { \l_keys_current_path_tl } {#1} {#2}
  }
}
%    \end{macrocode}
%\end{macro}
%\end{macro}
%    
%\begin{macro}{.cd:}
% The change-directory property simply alters the value of the default
% path. 
%    \begin{macrocode}
\keys_manage_quick:n {
    \c_keys_properties_path_tl /.cd:/.code:n = {
    \tl_set:Nx \l_keys_default_path_tl { \l_keys_current_path_tl / }
  }
}
%    \end{macrocode}
%\end{macro}
%
%\begin{macro}{.value_forbidden:}
%\begin{macro}{._required_bool}
%\begin{macro}{.value_required:}
%\begin{macro}{._forbidden_bool}
% Values are required or forbidden by creating the appropriate flags.
%    \begin{macrocode}
\keys_manage_quick:n {
  \c_keys_properties_path_tl /.value_required:/.code:n = {
    \keys_bool_new:n { \l_keys_current_path_tl /._required_bool }
    \keys_undefine:n { \l_keys_current_path_tl /._forbidden_bool }
  },
  \c_keys_properties_path_tl /.value_forbidden:/.code:n = {
    \keys_bool_new:n { \l_keys_current_path_tl /._forbidden_bool }
    \keys_undefine:n { \l_keys_current_path_tl /._required_bool }
  }
}
%    \end{macrocode}
%\end{macro}
%\end{macro}
%\end{macro}
%\end{macro}
%
%\begin{macro}{.default:n}
%\begin{macro}{._default_tl}
% The default value for a key is stored in the |._default_tl|
% private property.
%    \begin{macrocode}
\keys_manage_quick:n {
  \c_keys_properties_path_tl /.default:n/.code:n = {
    \keys_store:nn { \l_keys_current_path_tl /._default_tl } {#1}
    \keys_undefine:n { \l_keys_current_path_tl /._required_bool }
  }
}
%    \end{macrocode}
%\end{macro}
%\end{macro}
%
%\begin{macro}{.tl_set:N}
%\begin{macro}{.tl_set_x:N}
%\begin{macro}{.tl_gset:N}
%\begin{macro}{.tl_gset_x:N}
% Storage of the value in a tl, as given or expanded, local or global.
%    \begin{macrocode}
\keys_manage_quick:n {
  \c_keys_properties_path_tl /.tl_set:N/.code:n = {
    \keys_set:NN \tl_set:Nn #1
  },
  \c_keys_properties_path_tl /.tl_set_x:N/.code:n = {
    \keys_set:NN \tl_set:Nx #1
  },
  \c_keys_properties_path_tl /.tl_gset:N/.code:n = {
    \keys_set:NN \tl_gset:Nn #1
  },
  \c_keys_properties_path_tl /.tl_gset_x:N/.code:n = {
    \keys_set:NN \tl_gset:Nx #1
  }
}
%    \end{macrocode}
%\end{macro}
%\end{macro}
%\end{macro}
%\end{macro}
%
%\begin{macro}{.int_set:N}
%\begin{macro}{.int_gset:N}
%\begin{macro}{.skip_set:N}
%\begin{macro}{.skip_gset:N}
%\begin{macro}{.toks_set:N}
%\begin{macro}{.toks_gset:N}
% For int, skip and toks storage, no expansion to worry about.
%    \begin{macrocode}
\keys_manage_quick:n {
  \c_keys_properties_path_tl /.int_set:N/.code:n = {
    \keys_set:NN \int_set:Nn #1
  },
  \c_keys_properties_path_tl /.int_gset:N/.code:n = {
    \keys_set:NN \int_gset:Nn #1
  },
  \c_keys_properties_path_tl /.skip_set:N/.code:n = {
    \keys_set:NN \skip_set:Nn #1
  },
  \c_keys_properties_path_tl /.skip_gset:N/.code:n = {
    \keys_set:NN \skip_gset:Nn #1
  },
  \c_keys_properties_path_tl /.toks_set:N/.code:n = {
    \keys_set:NN \toks_set:Nn #1
  },
  \c_keys_properties_path_tl /.toks_gset:N/.code:n = {
    \keys_set:NN \toks_gset:Nn #1
  }
}
%    \end{macrocode}
%\end{macro}
%\end{macro}
%\end{macro}
%\end{macro}
%\end{macro}
%\end{macro}
%
% The prefix for the current module is needed when storing material
% by csname. 
%    \begin{macrocode}
\keys_manage_quick:n {
  \c_keys_utilities_path_tl /current_module:n/.code:n = {
    \tlist_if_empty:nTF {#1} {
      \tl_clear:N \l_keys_current_module_tl
    }{
      \tl_set:Nn \l_keys_current_module_tl { #1 _ }
    }
  }
}
%    \end{macrocode}
%    
%\begin{macro}{.tl_set:n}
%\begin{macro}{.tl_set_x:n}
%\begin{macro}{.tl_gset:n}
%\begin{macro}{.tl_gset_x:n}
%\begin{macro}{.int_set:n}
%\begin{macro}{.int_gset:n}
%\begin{macro}{.skip_set:n}
%\begin{macro}{.skip_gset:n}
%\begin{macro}{.toks_set:n}
%\begin{macro}{.toks_gset:n}
% With the module available, storage properties that only need the 
% unique part of the variable name are created.
%    \begin{macrocode}
\keys_manage_quick:n {
  \c_keys_properties_path_tl /.tl_set:n/.code:n = {
    \keys_set:Nnn \tl_set:cn { l_ } { #1 _tl }
  },
  \c_keys_properties_path_tl /.tl_set_x:n/.code:n = {
    \keys_set:Nnn \tl_set:cx { l_ } { #1 _tl }
  },
  \c_keys_properties_path_tl /.tl_gset:n/.code:n = {
    \keys_set:Nnn \tl_gset:cn { g_ } { #1 _tl }
  },
  \c_keys_properties_path_tl /.tl_gset_x:n/.code:n = {
    \keys_set:Nnn \tl_gset:cx { g_ } { #1 _tl }
  },
  \c_keys_properties_path_tl /.int_set:n/.code:n = {
    \keys_set:Nnn \int_set:cn { l_ } { #1 _int }
  },
  \c_keys_properties_path_tl /.int_gset:n/.code:n = {
    \keys_set:Nnn \int_gset:cn { g_ } { #1 _int }
  },
  \c_keys_properties_path_tl /.skip_set:n/.code:n = {
    \keys_set:Nnn \skip_set:cn { l_ } { #1 _skip }
  },
  \c_keys_properties_path_tl /.skip_gset:n/.code:n = {
    \keys_set:Nnn \skip_gset:cn { g_ } { #1 _skip }
  },
  \c_keys_properties_path_tl /.toks_set:n/.code:n = {
    \keys_set:Nnn \toks_set:cn { l_ } { #1 _toks }
  },
  \c_keys_properties_path_tl /.toks_gset:n/.code:n = {
    \keys_set:Nnn \toks_gset:cn { g_ } { #1 _toks }
  },
}
%    \end{macrocode}
%\end{macro}
%\end{macro}
%\end{macro}
%\end{macro}
%\end{macro}
%\end{macro}
%\end{macro}
%\end{macro}
%\end{macro}
%\end{macro}
%
%\begin{macro}{.bool_set:N}
%\begin{macro}{.bool_gset:N}
%\begin{macro}{.bool_set:n}
%\begin{macro}{.bool_gset:n}
% The properties for switches look similar, although internally things
% are rather different.
%    \begin{macrocode}
\keys_manage_quick:n {
  \c_keys_properties_path_tl /.bool_set:N/.code:n = {
    \keys_bool_set:nN { set } #1
  },
  \c_keys_properties_path_tl /.bool_gset:N/.code:n = {
    \keys_bool_set:nN { gset } #1
  },
  \c_keys_properties_path_tl /.bool_set:n/.code:n = {
    \keys_bool_set:nnn { set } { l_ } { #1 }
  },
  \c_keys_properties_path_tl /.bool_gset:n/.code:n = {
    \keys_bool_set:nnn { gset } { g_ } { #1 }
  },
}
%    \end{macrocode}
%\end{macro}
%\end{macro}
%\end{macro}
%\end{macro}
%
%\begin{macro}{.bool_set_inverse:N}
%\begin{macro}{.bool_gset_inverse:N}
%\begin{macro}{.bool_set_inverse:n}
%\begin{macro}{.bool_gset_inverse:n}
% The inverse versions of the switches are handled in the same way
% here.
%    \begin{macrocode}
\keys_manage_quick:n {
  \c_keys_properties_path_tl /.bool_set_inverse:N/.code:n = {
    \keys_bool_set_inverse:nN { set } #1
  },
  \c_keys_properties_path_tl /.bool_gset_inverse:N/.code:n = {
    \keys_bool_set_inverse:nN { gset } #1
  },
  \c_keys_properties_path_tl /.bool_set_inverse:n/.code:n = {
    \keys_bool_set_inverse:nnn { set } { l_ } { #1 }
  },
  \c_keys_properties_path_tl /.bool_gset_inverse:n/.code:n = {
    \keys_bool_set_inverse:nnn { gset } { g_ } { #1 }
  },
}
%    \end{macrocode}
%\end{macro}
%\end{macro}
%\end{macro}
%\end{macro}
%
%\begin{macro}{.expects_choice:}
% Multiple choice keys are created by searching sub-keys. So the
% code to make a key into a multiple choice is quite simple.
%    \begin{macrocode}
\keys_manage_quick:n {
  \c_keys_properties_path_tl /.expects_choice:/.code:n = {
    \keys_manage_internal:n {
      \l_keys_current_path_tl /.cd:,
      .code:n = {
        \tl_set:Nn \l_keys_current_choice_tl {##1}
        \int_zero:N \l_keys_current_choice_int
        \exp_args:No \keys_parse_list:n 
          { \l_keys_current_key_full_tl / ##1 }
      },
      unknown/.code:n = {
        \keys_error:nn { unknown~choice } 
          { \l_keys_current_key_name_tl }
      }
    }
  },
}
%    \end{macrocode}
%\end{macro}
%\begin{macro}{.create_choices:nn}
%\begin{macro}{.create_choices:nx}
% Creating choices as a block.
%    \begin{macrocode}
\keys_manage_quick:n {
  \c_keys_properties_path_tl /.create_choices:nn/.code:Nn = 2 {
    \keys_choices_create:Nnn n {#1} {#2}
  },
  \c_keys_properties_path_tl /.create_choices:nx/.code:Nn = 2 {
    \keys_choices_create:Nnn x {#1} {#2}
  },
}
%    \end{macrocode}
%\end{macro}
%\end{macro}
%
%\begin{macro}{.use_keys:n}
%\begin{macro}{.use_keys:x}
%\begin{macro}{.use_keys:Nn}
%\begin{macro}{.use_keys:Nx}
% Keys calling other keys is actually quite easy.
%    \begin{macrocode}
\keys_manage_quick:n {
  \c_keys_properties_path_tl /.use_keys:n/.code:n = {
    \keys_manage_internal:n {
      \l_keys_current_path_tl /.code:n = { \keys_parse_list:n {#1} }
    }
  },
  \c_keys_properties_path_tl /.use_keys:x/.code:n = {
    \keys_manage_internal:n {
      \l_keys_current_path_tl /.code:x = {
        \exp_not:N \keys_parse_list:n {#1}
      }
    }
  },
  \c_keys_properties_path_tl /.use_keys:Nn/.code:Nn = 2 {
    \keys_manage_internal:n {
      \l_keys_current_path_tl /.code:Nn = #1 { \keys_parse_list:n {#2} }
    }
  },
  \c_keys_properties_path_tl /.use_keys:Nx/.code:Nn = 2 {
    \keys_manage_internal:n {
      \l_keys_current_path_tl /.code:Nx =
        #1 { \exp_not:N \keys_parse_list:n {#2} }
    }
  }
}
%    \end{macrocode}
%\end{macro}
%\end{macro}
%\end{macro}
%\end{macro}
%
%\begin{macro}{.try:n}
%\begin{macro}{.retry:n}
% For attempting to set keys without assuming they exist.
%    \begin{macrocode}
\keys_manage_quick:n {
  \c_keys_properties_path_tl /.try:n/.code:n = { \keys_try: },
  \c_keys_properties_path_tl /.retry:n/.code:n = {
    \bool_if:NF \l_keys_success_bool { \keys_try: }
  }
}
%    \end{macrocode}
%\end{macro}
%\end{macro}
%
%\begin{macro}{.show_code:}
%\begin{macro}{.show_key:}
% Finally, two keys for debugging problems.
%    \begin{macrocode}
\keys_manage_quick:n {
  \c_keys_properties_path_tl /.show_code:/.code:n = {
    \cs_show:c { 
      \c_keys_cs_prefix_tl \l_keys_current_path_tl /._cmd:w 
    }
  },
  \c_keys_properties_path_tl /.show_key:/.code:n = {
    \cs_show:c { \c_keys_cs_prefix_tl \l_keys_current_path_tl }
  }
}
%</package>
%    \end{macrocode}
%\end{macro}
%\end{macro}
%
%\Finale
