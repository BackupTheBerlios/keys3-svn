%\iffalse
%% File: keys3.dtx Copyright (C) 2008 Joseph Wright
%%
%% It may be distributed and/or modified under the conditions of the
%% LaTeX Project Public License (LPPL), either version 1.3c of this
%% license or (at your option) any later version.  The latest version
%% of this license is in the file
%%
%%    http://www.latex-project.org/lppl.txt
%%
%% -----------------------------------------------------------------------
%<*driver|package>
\RequirePackage{l3names}
%</driver|package>
%\fi
\GetIdInfo$Id$
  {Key management for LaTeX3}
%\iffalse
%<*driver>
%\fi
\ProvidesFile{\filename.\filenameext}
  [\filedate\space v\fileversion\space\filedescription]
%\iffalse
\documentclass[full]{l3doc}
\begin{document}
\DocInput{\filename.\filenameext}
\end{document}
%</driver>
% \fi
%
% \title{The \textsf{keys3} package^^A
%   \thanks{This file has version number \fileversion, last
%     revised \filedate.}\\
% Key management for \LaTeX3}
% \author{Joseph Wright^^A
%   \thanks{E-mail: joseph.wright@morningstar2.co.uk}}
% \date{\filedate}
% \maketitle
%
%\section{Key management}
%
% The key--value method is a popular system for creating large numbers
% of settings for controlling macro or package behaviour.  For the
% user, the system normally results in input of the form
%\begin{verbatim}
%  \PackageControlMacro{
%    key     = value,
%    key two = value two
%  }
%\end{verbatim}
% or
%\begin{verbatim}
%  \PackageMacro[
%    key     = value,
%    key two = value two
%  ]{argument}.
%\end{verbatim}
% For the programmer, the original \textsf{keyval} package gives only
% the most basic interface for this work.  All key macros have to be
% created one at a time, and as a result the \textsf{kvoptions} and
% \textsf{xkeyval} packages have been written to extend the ease of
% creating keys.  However, the underlying model is rather inflexible.
%
% As an alternative, \textsf{pgfkeys} from the \textsf{pgf}/Ti\emph{k}z
% bundle uses a ``file-like'' model for keys. In this model, each key
% has one or more functions attached to define its action.  In
% \textsf{pgfkeys} terminology, these are key \emph{handlers}. Keys can
% be created and used using a unified interface:
%\begin{verbatim}
%  \pgfkeys{
%    /path/key/.handler = code,
%    /path/key          = value
%  }
%\end{verbatim}
%
% The \textsf{keys3} package is aimed at creating a high-level
% programmers interface for key--value controls in \LaTeX3. Key
% creation and control follows the \textsf{pgfkeys} closely, although
% changes have been made to adhere to the new coding ideas of \LaTeX3,
% and new functions have been added.  The \textsf{keys3} package is
% \emph{not} a simple translation of \textsf{pgfkeys} to the new syntax.
% In package internals have been written from the ground up, to better
% enforce variable typing and to act in a know manner under failing
% circumstances.
%
% In the \textsf{keys3} model, each key belongs to a module,
% which may have one or more subdivisions. Each key then has
% one or more properties which define how the key acts. The idea
% of a single function for all key control is retained:
%\begin{verbatim}
%  \keys_manage:n{
%    /module/key/.property     = code,
%    /module/sub/key/.property = more code,
%    /module/key               = value
%  }
%\end{verbatim}
% The combination of all \m{module} parts up to the \m{key} is referred
% to as the \emph{path} of the key.
%
%\subsection{Creating, retrieving and setting keys}
%
% The main interface for key management is the the \cs{keys_manage:n}
% macro.  This can be used to create, retrieve and set keys, and is
% therefore the preferred interface for the package.
%
% \begin{function}{%
%   \keys_manage:n|
%   \keys_manage_quick:n}
%   \begin{syntax}
%     "\keys_manage:n" \Arg{keyval list}
%   \end{syntax}
%   The main key management function, which parses over the <keyval
%   list> and processes each key. The standard version removes leading
%   and trailing spaces and checks catcodes for ungrouped ``|,|'' and
%   ``|=|''. The |_quick| version omits these tests for speed, and is
%   therefore more suited to creating keys inside code blocks.
% \end{function}
%
% \begin{function}{%
%   \keys_manage:nn|
%   \keys_manage_quick:nn}
%   \begin{syntax}
%     "\keys_manage:nn" \Arg{path} \Arg{keyval list}
%   \end{syntax}
%   Versions of the above which include a <path> as an argument. These
%   are faster versions of the |:n| macros when setting lots of keys
%   separately and with known paths.
% \end{function}
%
% Setting up and altering keys is carried out using one or more
% properties (in \textsf{pgfkeys}, these are called handlers). In all
% cases, \m{key} may be a full-qualified key with a path, or a partial
% key to which the default path will be added. Notice that when a
% single argument is required, the braces around the argument may be
% omitted without any error arising.
% 
% \begin{function}{%
%   /.apply~keys:n|
%   /.apply~keys:x|
%   /.apply~keys:nn|
%   /.apply~keys:nx}
%   \begin{syntax}
%     <key>/.apply~keys:n = \Arg{keyval list}
%   \end{syntax}
%   Defines a set of keys to be applied when <key> is called. The usual
%   parameter |#1| is available.
%   \begin{texnote}
%     These are the \textsf{pgfkeys} |.style| handler and relatives.
%  \end{texnote}
% \end{function}
%
% \begin{function}{%
%   /.boolean:N|
%   /.boolean:c|
%   /.gboolean:N|
%   /.gboolean:c}
%   \begin{syntax}
%     <key>/.boolean:N = <Boolean>
%     <key>/.boolean:c = \Arg{name}
%   \end{syntax}
%   Creates <Boolean> switch, and uses the <key> to set the Boolean to
%   |true| or |false|. Other input raises an error, but with a default
%   value of |true| set. The behaviour with unknown values is altered
%   by adding code to \texttt{<key>/unknown}. The |:c| variant uses a
%   switch \cs{l_<module>_<name>_bool}, with <module> set as described
%   in secion~\ref{sec:keys:prog}. The |g| variants set the flag
%   globally.
% \end{function}
%
% \begin{function}{/.cd:}
%   \begin{syntax}
%     <key>/.cd:
%   \end{syntax}
%   Change the current path to <key>. Usually the first property used
%   when setting up several keys.
% \end{function}
%
% \begin{function}{/.choice:}
%   \begin{syntax}
%     <key>/.choice:
%   \end{syntax}
%   Makes <key> a choice key.  Each possible choice should be stored as
%   a sub-key of <key>, \emph{i.e.} <key>/<choice>, and should have and
%   action attached (|.code:n|, |.apply keys:n|, \emph{etc.}). The 
%   behaviour with unspecified values is altered by adding code to the
%   \texttt{<key>/unknown} sub-key; initially, this raises an error if
%   an unknown value is given.
% \end{function}
%
% \begin{function}{/.choices:nn}
%   \begin{syntax}
%     <key>/.choices:nn = \Arg{choice list} \Arg{code}
%   \end{syntax}
%   For a comma-separated <choice list>, creates each <choice> as a
%   sub-key of the current <key>. The <code> is associated with each
%   <choice> as if |.code:n| had been run on each one. The <code> can
%   access the choice made \emph{via} \cs{l_keys_current_choice_tlp}.
%   The position of the <choice> in the <choice list> is available as
%   \cs{l_keys_current_choice_int}.
%   \begin{texnote}
%     The role of \cs{l_keys_current_choice_tlp} and
%     \cs{l_keys_current_choice_int} is similar to \cs{val} and \cs{nr}
%     in the \textsf{xkeyval} manual for the description of
%     \cs{define@chociekey}.
%   \end{texnote}
% \end{function}
%
% \begin{function}{%
%   /.code:n|
%   /.code:x}
%   \begin{syntax}
%     <key>/.code:n = \Arg{code}
%   \end{syntax}
%   Sets <key> to execute <code> when the key is called. The <code>
%   will be passed a single argument |#1|. This property clears any
%   others previously set for the key.
% \end{function}
%
% \begin{function}{%
%   /.code:nn|
%   /.code:nx}
%   \begin{syntax}
%     <key>/.code:nn = \Arg{parameters} \Arg{code}
%   \end{syntax}
%   Sets <key> to execute <code> when the key is called. The function
%   used will have \TeX\ <parameters>.
% \end{function}
%
% \begin{function}{/.default:n}
%   \begin{syntax}
%     <key>/.default:n = \Arg{value}
%   \end{syntax}
%   Use <value> whenever <key> is given without a value.
% \end{function}
%
% \begin{function}{/.equivalent~key:n}
%   \begin{syntax}
%     <key>/.equivalent~key:n = <key2>
%   \end{syntax}
%   Makes <key2> a key with equivalent meaning to <key>: intended to
%   make creating alternative spellings or choices simple.
% \end{function}
% 
% \begin{function}{.initial:n}
%   \begin{syntax}
%     <key>/.initial:n = \Arg{value}
%   \end{syntax}
%   Initialise <key> with <value>, until a second assignment is made.
% \end{function}
%
% \begin{function}{/.inverse~key:n}
%   \begin{syntax}
%     <key>/.inverse key:n = \Arg{name}
%   \end{syntax}
%   Creates a new key <name> as an inverse of existing Boolean <key>.
%   The inverse key will work in the opposite sense to the parent. Thus
%   setting <name> to \texttt{true} is the same as setting <key> to
%   \texttt{false}.
% \end{function}
%
% \begin{function}{%
%   .key~to~tlp:N|
%   .key~to~tlp:c}
%   \begin{syntax}
%     <key>/.key~to~tlp:N = <tlp>
%   \end{syntax}
%   Set <tlp> equal to the current contents of <key>.
% \end{function}
%
% \begin{function}{%
%   .let~to~func:N|
%   .let~to~func:c}
%   \begin{syntax}
%     <key>/.let~to~func:N = <function>
%   \end{syntax}
%   When <key> is called, the argument will be \cs{let:NN} to
%   <function>. The \texttt{:c} variant creates a function 
%   \cs{<module>_<function>:w}, using the current <module> (if set)
%   in the same way as the \texttt{.store~in} keys. Note that in this
%   case the function created has a \texttt{:w} argument specifier, as
%   \textsf{keys3} does not have any clue as to the arguments required.
% \end{function}
%
% \begin{function}{/.module:}
%   \begin{syntax}
%     <key>/.module:
%   \end{syntax}
%   Declares <key> as a module name: giving <key> alone is then the
%   same as giving <key>/.cd:, but is a little faster.
% \end{function}
%
% \begin{function}{%
%   /.put~left:n|
%   /.put~right:n}
%   \begin{syntax}
%     <key>/.put left:n = \Arg{value}
%   \end{syntax}
%   Adds <value> to data stored in <key>.
% \end{function}
%
% \begin{function}{%
%   /.put~code~left:n|
%   /.put~code~right:n}
%   \begin{syntax}
%     <key>/.put code left:n = \Arg{code}
%   \end{syntax}
%   Adds <code> to <key> macro created by |.code:n|.
% \end{function}
%
% \begin{function}{%
%   /.put~keys~left:n|
%   /.put~keys~right:n}
%   \begin{syntax}
%     <key>/.put keys left:n = \Arg{keyval list}
%   \end{syntax}
%   Adds <keyval list> to <key> code created by |.apply~keys:n|.
% \end{function}
%
% \begin{function}{%
%   /.show~code:|
%   /.show:}
%   \begin{syntax}
%     <key>/.show code:
%     <key>/.show:
%   \end{syntax}
%   Show the contents of the key macro itself or the hidden property
%   storing |.code:n|. May be useful for debugging, but note that all
%   |._cmd:w| keys have a delimited argument ending \cs{q_keys_stop}.
% \end{function}
%
%
% \begin{function}{%
%   /.store~in~int:N|
%   /.store~in~int:c|
%   /.gstore~in~int:N|
%   /.gstore~in~int:c}
%   \begin{syntax}
%     <key>/.store~in~int:N = <int>
%     <key>/.store~in~int:c = \Arg{name}
%   \end{syntax}
%   When <key> is called, store the value in <int>. No check is made on the
%   <int> at definition time: when the key is used, the value is stored
%   with no checks. For the |:c| variant, an int of name
%   \cs{l_<module>_<name>_int} is used for the storage. The <module>
%   value is defined by |/keys/.current module:n| as described in
%   section~\ref{sec:keys:prog}. The |g| variants are global.
% \end{function}
% 
% \begin{function}{%
%   /.store~in~skip:N|
%   /.store~in~skip:c|
%   /.gstore~in~skip:N|
%   /.gstore~in~skip:c}
%   \begin{syntax}
%     <key>/.store~in~skip:N = <skip>
%     <key>/.store~in~skip:c = \Arg{name}
%   \end{syntax}
%   When <key> is called, store the value in <skip>. No check is made on
%   the <skip> at definition time: when the key is used, the value is 
%   stored with no checks. For the |:c| variant, a skip of name
%   \cs{l_<module>_<name>_skip} is used for the storage. The <module>
%   value is defined by |/keys/.current module:n| as described in
%   section~\ref{sec:keys:prog}. The |g| variants are global.
% \end{function}
%
% \begin{function}{%
%   /.store~in~tlp:N|
%   /.store~in~tlp:c|
%   /.xstore~in~tlp:N|
%   /.xstore~in~tlp:c|
%   /.gstore~in~tlp:N|
%   /.gstore~in~tlp:c|
%   /.xgstore~in~tlp:N|
%   /.xgstore~in~tlp:c}
%   \begin{syntax}
%     <key>/.store~in~tlp:N = <tlp>
%     <key>/.store~in~tlp:c = \Arg{name}
%   \end{syntax}
%   When <key> is called, store the value in <tlp>. The |xstore| variant
%   expands the value at the point of storage. No check is made on the
%   <tlp> at definition time: when the key is used, the value is stored
%   with no checks. For the |:c| variant, a tlp of name
%   \cs{l_<module>_<name>_tlp} is used for the storage. The <module>
%   value is defined by |/keys/.current module:n| as described in
%   section~\ref{sec:keys:prog}. The |g| variants are global, while the
%   |x| variants expand the content given before storing.
% \end{function}
% 
% \begin{function}{%
%   /.store~in~toks:N|
%   /.store~in~toks:c|
%   /.gstore~in~toks:N|
%   /.gstore~in~toks:c}
%   \begin{syntax}
%     <key>/.store~in~toks:N = <toks>
%     <key>/.store~in~toks:c = \Arg{name}
%   \end{syntax}
%   When <key> is called, store the value in <toks>. No check is made on 
%   the <int> at definition time: when the key is used, the value is 
%   stored  with no checks. For the |:c| variant, a toks of name
%   \cs{l_<module>_<name>_toks} is used for the storage. The <module>
%   value is defined by |/keys/.current module:n| as described in
%   section~\ref{sec:keys:prog}. The |g| variants are global.
% \end{function}
%
% \begin{function}{%
%   .tlp~to~key:N|
%   .tlp~to~key:c}
%   \begin{syntax}
%     <key>/.tlp~to~key:N = <tlp>
%   \end{syntax}
%   Set <key> equal to the current contents of <tlp>.
% \end{function}
%
% \begin{function}{%
%   /.try:n|
%   /.retry:n}
%   \begin{syntax}
%     <key>/.try:n = \Arg{value}
%   \end{syntax}
%   Attempts to set <key> with <value>, with no errors occurring if the
%   <key> is not defined.  |.retry:n| only attempts to set the key if
%   a previous |.try:n| has failed.
% \end{function}
%
%\subsubsection{Programmers keys}
%\label{sec:keys:prog}
%
% As well as the pre-defined property and error keys, the key
% |/keys/execute:n| is available to run arbitrary code. This can be
% used to carry out any necessary definitions without leaving
% \cs{keys_manage:n}.
%
% The key |/keys/current module:n| is used to set the name of the
% current module for use with properties |.boolean:c| and
% \texttt{.store in \ldots}. If this key has been set, the module 
% name is used when creating |_bool| and |_tlp| storage.
%\begin{verbatim}
%  \keys_manage:n {
%    /module/.cd:,
%    key one/.boolean:c       = one,    % Uses \l_one_bool
%    key two/.store in tlp:c  = two,    % Uses \l_two_tlp
%    /keys/current module:n   = module,
%    key three/.boolean:c     = three,  % Uses \l_module_three_bool
%    key four/.store in tlp:c = four,   % Uses \l_module_four_tlp
%  }
%\end{verbatim}
% As illustrated, an additional ``|_|'' is included in the name when a
% module name is available.  The current module name is reset to no
% value for each call to \cs{keys_manage:n}.
%
%\subsection{Creating choices}
%
% Choices can be created using either the |.choice:| or |.choices:nn|
% properties.  The first of these makes the current key a choice, and
% leaves it to the programmer to create each choice key separately. This
% is therefore best suited when each choice is rather different.
%\begin{verbatim}
%  \keys_manage:n {
%    /module/.cd:
%    key/.choice:,
%    key/.cd:,
%      egg/.code:n        = <some code>,
%      cheese/.code:x     = <some expanded code>,
%      fish/.apply keys:n = <some extra keys>
%  }
%\end{verbatim}
%
% On the other hand, the |.choices:nn| property is intended to be
% useful when the choices are rather similar.  The tlp
% \cs{l_keys_current_choice_tlp}, which contains the choice name, can
% be used in the construction of csnames, \emph{etc}.
%\begin{verbatim}
%  \keys_manage:n {
%    /module/.cd:
%    key/.choices:nn =
%      {egg,cheese,fish}{You asked for \l_keys_current_choice_tlp!}
%  }
%\end{verbatim}
%
% Extra keys can be added to those generated by |.choices:nn| by creating
% sub-keys of the choice key. Unknown values can also be handled using
% the \texttt{unknown} choice.
%\begin{verbatim}
%  \keys_manage:n {
%    /module/.cd:
%    key/.choices:nn =
%      {egg,cheese,fish}{You asked for \l_keys_current_choice_tlp!},
%    key/unknown/.code:n =
%      {I don't known \l_keys_current_choice_tlp}
%  }
%\end{verbatim}
%
%\subsection{Functions}
%
%\subsubsection{Public functions}
%
% Although \cs{keys_manage:n} can carry out all of the functions of the
% package, some of the internal macros may be of more general use for
% creating complex systems. A small number of non-specific macros are
% also created.
%
% \begin{function}{%
%   \quark_if_value_required:NTF|
%   \quark_if_value_required:NT|
%   \quark_if_value_required:NF}
%   \begin{syntax}
%     "\quark_if_value_required:NTF"
%     ~~~~<token> \Arg{true code} \Arg{false code}
%   \end{syntax}
%   Tests if <token> is equal to \cs{q_value_required}. The only
%   functions which do not start with \cs{keys}.
% \end{function}
%
% \begin{function}{%
%   \keys_if_really_exist:cTF|
%   \keys_if_really_exist:cT|
%   \keys_if_really_exist:cF}
%   \begin{syntax}
%     "\keys_if_really_exist:cTF"
%     ~~~~\Arg{key} \Arg{true code} \Arg{false code}
%   \end{syntax}
%   Tests if <key> is known without adding anything to the hash table.
% \end{function}
%
% In all cases, \m{key} should be a fully-qualified key name starting
% with |/| and including the \m{module}, \emph{etc}.
% 
% \begin{function}{\keys_manage_internal:n}
%   \begin{syntax}
%     "\keys_manage_internal:n" \Arg{keyval list}
%   \end{syntax}
%   Private version of \cs{keys_manage:n} for internal use.
% \end{function}
%
% \begin{function}{\keys_process:n}
%   \begin{syntax}
%     "\keys_process:n" \Arg{keyval list}
%   \end{syntax}
%   Common macro used by all key management functions to initiate key
%   evaluation. Can be used to recursively set keys (see for example
%   definition of |.apply keys:n|).
% \end{function}
%
% \begin{function}{%
%   \keys_def:cn|
%   \keys_def:co|
%   \keys_def:cx}
%   \begin{syntax}
%     "\keys_def:cn" \Arg{key} \Arg{value}
%   \end{syntax}
%   Stores <value> in <key>: <value> should be a tlp-compatible
%   argument.
% \end{function}
%
% \begin{function}{%
%   \keys_use:c}
%   \begin{syntax}
%     "\keys_use:c" \Arg{key}
%   \end{syntax}
%   Use the code stored in <key>.
% \end{function}
%
%\subsubsection{Internal functions}
%
% \begin{function}{\keys_bool_new:c}
%   \begin{syntax}
%     "\keys_bool_new:c" \Arg{key}
%   \end{syntax}
%   Create a new Boolean <key> and sets to |true|.
% \end{function}
%
% \begin{function}{%
%   \keys_def_cmd:cn|
%   \keys_def_cmd:cx}
%   \begin{syntax}
%     "\keys_def_cmd:cn" \Arg{key} \Arg{code}
%   \end{syntax}
%   Creates the internal |._cmd:w| propery  for <key>, with <code> as
%   the definition and accepting one argument |#1|.
% \end{function}
%
% \begin{function}{%
%   \keys_def_cmd:cnn|
%   \keys_def_cmd:cnx}
%   \begin{syntax}
%     "\keys_def_cmd:cnn" \Arg{key} \Arg{parameters} \Arg{code}
%   \end{syntax}
%   Creates the internal |._cmd:w| property  for <key>, with <code> as
%   the definition and accepting arbitrary parameters as specified in
%   <parameters>.
% \end{function}
%
% \begin{function}{\keys_choice_create:n}
%   \begin{syntax}
%     "\keys_choice_create:n" \Arg{choice}
%   \end{syntax}
%   Creates a choice key from a list.
% \end{function}
%
% \begin{function}{\keys_clear_properties:c}
%   \begin{syntax}
%     "\keys_clear_properties:c" \Arg{key}
%   \end{syntax}
%   Removes all existing internal properties from <key>.
% \end{function}
%
% \begin{function}{%
%   \keys_if_cmd_really_exist:cTF|
%   \keys_if_cmd_really_exist:cT|
%   \keys_if_cmd_really_exist:cF}
%   \begin{syntax}
%     "\keys_if_cmd_really_exist:cTF"
%     ~~~~\Arg{key} \Arg{true code} \Arg{false code}
%   \end{syntax}
%   Tests if <key> |._cmd:w| property exists.
% \end{function}
%
% \begin{function}{\keys_expand:n}
%   \begin{syntax}
%     "\keys_expand:n" \Arg{value}
%   \end{syntax}
%   Helper macro for expansion control.
% \end{function}
%
% \begin{function}{\keys_let:cc}
%   \begin{syntax}
%     "\keys_let:cc" \Arg{key1} \Arg{key2}
%   \end{syntax}
%   Let <key1> to <key2>.
% \end{function}
%
% \begin{function}{%
%   \keys_manage_aux:nn|
%   \keys_manage_aux:nnn}
%   \begin{syntax}
%     "\keys_manage_aux:nn" \Arg{current path} \Arg{keyval list}
%     "\keys_manage_aux:nnn"
%     ~~~~\Arg{current path} \Arg{new path} \Arg{keyval list}
%   \end{syntax}
%   Auxiliary function for \cs{keys_manage:n} and \cs{keys_manage:nn},
%   respectively, used to preserve path information through nested
%   calls.
% \end{function}
%
% \begin{function}{\keys_module_name:}
%   \begin{syntax}
%     "\keys_module_name:"
%   \end{syntax}
%   Sets \cs{l_keys_tmpa_tlp} to hold the appropriate name for
%   creating Boolean and tlp functions.
% \end{function}
%
% \begin{function}{%
%   \keys_path_add:w|
%   \keys_path_add_aux:w}
%   \begin{syntax}
%     "\keys_path_add:w" <key> "\q_stop"
%   \end{syntax}
%   Adds a path to <key> if necessary.
% \end{function}
%
% \begin{function}{\keys_parse:n}
%   \begin{syntax}
%     "\keys_parse:n" \Arg{keyval list}
%   \end{syntax}
%   The function to actually parse the user input.  Is \cs{let:NN} to
%   one from \textsf{l3keyval}.
% \end{function}
%
% \begin{function}{%
%   \keys_process_elt:nn|
%   \keys_process_elt:n|
%   \keys_process_elt:}
%   \begin{syntax}
%     "\keys_process_elt:nn" \Arg{key} \Arg{value}
%   \end{syntax}
%   Receives <key> and <value> from \textsf{l3keyval} and carries out
%   processing.  The |:n| variant is used to process keys with no
%   <value>. The |:| version is used internally once some initial
%   testing of the key--value combination has occurred.
% \end{function}
%
% \begin{function}{\keys_put:ncn}
%   \begin{syntax}
%     "\keys_put:ncn" \Arg{position} \Arg{key} \Arg{data}
%   \end{syntax}
%   Add <data> to the <position> of the value stored in <key>.
% \end{function}
%
% \begin{function}{\keys_put_code:nc}
%   \begin{syntax}
%     "\keys_put_code:nc" \Arg{position} \Arg{key}
%   \end{syntax}
%   Adds <value> to <position> of <key> |._cmd:w| macro.
% \end{function}
%
% \begin{function}{%
%   \keys_separate:|
%   \keys_separate:w}
%   \begin{syntax}
%     "\keys_separate:"
%     "\keys_separate:w" <key> / "\q_stop"
%   \end{syntax}
%   Separates <key> into a <path> and a <name>.
% \end{function}
% 
% \begin{function}{%
%   \keys_store:NN|
%   \keys_store:NN}
%   \begin{syntax}
%     "\keys_store:NN" <func> <store>
%   \end{syntax}
%   Used to store argument of <key> using <func> into <store>.
% \end{function}
%
% \begin{function}{\keys_toks_new:cn}
%   \begin{syntax}
%     "\keys_toks_new:cn" \Arg{key} \Arg{tokens}
%   \end{syntax}
%   Create a new toks <key> and store <tokens>.
% \end{function}
%
% \begin{function}{\keys_toks_set:Nc}
%   \begin{syntax}
%     "\keys_toks_set:Nc" <toks> \Arg{key}
%   \end{syntax}
%   Transfers the contents of <key> into <toks>.  Used for manipulation
%   of the toks containing the current value. The <value> of <key>
%   should be expandable.
% \end{function}
%
% \begin{function}{%
%   \keys_use_cmd:cn|
%   \keys_use_cmd:c}
%   \begin{syntax}
%     "\keys_use_cmd:cn" \Arg{key} \Arg{argument}
%   \end{syntax}
%   Uses the internal |._cmd:w| key needed to execute the function of
%   <key> on <argument>. The |.c| variant uses the current value,
%   replacing it with an empty argument if it is equal to
%   \cs{q_no_value}.
% \end{function}
%
% Some functions are related to error-message creation.
%
% \begin{function}{\keys_err_new:cNnnn}
%   \begin{syntax}
%     "\keys_err_new:cNnnn" \Arg{error} <args>
%     ~~~~\Arg{short text} \Arg{long text} \Arg{code}
%   \end{syntax}
%   Creates error message <error> taking <args> arguments, with text as
%   specified and recovery <code>.
% \end{function}
%
% \begin{function}{\keys_err_use:cw}
%   \begin{syntax}
%     "\keys_err_use:cw" \Arg{error}
%   \end{syntax}
%   Use <error> with any arguments required.
% \end{function}
%
%\subsection{Variables and constants}
%
% \begin{variable}{\c_keys_cs_prefix_tlp}
%   Constant that contains the prefix for the csnames used for all
%   keys.
%   \begin{texnote}
%     The key macros are all of the form \cs{keys/<path>/<key>}, where
%     <path> can be equal to <module> or may be more complex. The
%     <property> keys are of form \cs{keys/<path>/<key>/.<property>}.
%   \end{texnote}
% \end{variable}
%
% \begin{variable}{\c_keys_root_tlp}
%   Constant that contains the ``root path'' for keys.
% \end{variable}
%
% \begin{variable}{%
%   \c_keys_errors_path_tlp|
%   \c_keys_properties_path_tlp|
%   \c_keys_utilities_path_tlp}
%   The paths of the error handling, property and utility keys.
% \end{variable}
%
% \begin{variable}{\c_keys_err_tlp}
%   Constant that contains the name of the error file in use.
% \end{variable}
%
% \begin{variable}{%
%   \l_keys_current_path_tlp|
%   \l_keys_default_path_tlp|
%   \l_keys_choice_path_tlp}
%   The values of the current and default search paths, plus a saved
%   value if a choice is being created.
% \end{variable}
%
% \begin{variable}{%
%   \l_keys_current_key_tlp|
%   \l_keys_current_name_tlp}
%   The values of current fully qualified key name (including path) and
%   key name alone.
% \end{variable}
%
% \begin{variable}{%
%    \l_keys_current_value_toks|
%    \l_keys_current_value_tlp}
%   The current value of the key.
%   \begin{texnote}
%     The <value> of the key might contain anything. This is stored as
%     a toks, but there are places where a single token test is needed:
%     for that a tlp is also created. The tlp is never used for
%     anything else: it is potentially ``funny''.
%   \end{texnote}
% \end{variable}
%
% \begin{variable}{\l_keys_current_module_tlp}
%   The name of the current module.
% \end{variable}
%
% \begin{variable}{%
%   \l_keys_current_choice_tlp|
%   \l_keys_current_choice_int}
%   Marker information for choices.
% \end{variable}
%
% \begin{variable}{%
%   \q_keys_stop|
%   \q_value_required}
%   Two new quarks are required.
%   \begin{texnote}
%     The \cs{q_value_required} quark is available generally to
%     indicate a value is needed for something, and has not been given.
%      On the other hand, \cs{q_keys_stop} is private to this package:
%     do not use it elsewhere.
%   \end{texnote}
% \end{variable}
%
% \begin{variable}{%
%   \l_keys_args_toks|
%   \l_keys_body_toks}
%   Private storage areas used to add code to an existing key.
% \end{variable}
%
% \begin{variable}{%
%   \l_keys_err_boolean_expected_tlp|
%   \l_keys_err_def_x_args_tlp|
%   \l_keys_err_not_boolean_tlp|
%   \l_keys_err_unknown_choice_tlp|
%   \l_keys_err_unknown_key_tlp|
%   \l_keys_err_value_ignored_tlp|
%   \l_keys_err_value_required_tlp}
%   Error message identifiers.
% \end{variable}
%
% \begin{variable}{\l_keys_success_bool}
%   Boolean for success.
% \end{variable}
%
% \begin{variable}{\l_keys_tmpa_tlp}
%   A scratch tlp.
% \end{variable}
%
%\subsection{Internal properties}
%
% The private key properties are used internally by \textsf{keys3}.
% They should not be addressed directly except from within the package.
% All private properties contain ``|_|'' before the property name as a
% reminder that they are internal.
%
% \begin{function}{/._args_toks}
%   \begin{syntax}
%   \end{syntax}
%   Stores the arguments of a |.code:nn| or |.code:nx| property for use
%   with |.code put left:n| and |.code put right:n|.
% \end{function}
%
% \begin{function}{/._body_toks}
%   \begin{syntax}
%   \end{syntax}
%   Stores the body of a |.code:nn| or |.code:nx| property for use with
%   |.code put left:n| and |.code put right:n|.
% \end{function}
%
% \begin{function}{/._boolean}
%   \begin{syntax}
%   \end{syntax}
%   Stores the name of the switch used by a Boolean key, and also
%   indicates that the <key> is a Boolean.
% \end{function}
%
% \begin{function}{/._cmd:w}
%   \begin{syntax}
%   \end{syntax}
%   Executes code stored for <key>.  The argument of the function is
%   delimited by \cs{q_keys_stop}.
% \end{function}
%
% \begin{function}{/._default_tlp}
%   \begin{syntax}
%   \end{syntax}
%   Stores the default value for <key>.
% \end{function}
%
% \begin{function}{/._forbidden_bool}
%   \begin{syntax}
%   \end{syntax}
%   Indicates that a value is forbidden for <key>.
% \end{function}
%
%\subsection{To think about}
%
% All keys created by \textsf{keys3} are case-sensitive. This is
% particularly noticeable for choice keys, where \textsf{xkeyval}
% provides both case-sensitive and case-insensitive variants.  Is there
% a need for both types of choice, or is the convention that everything
% is sensitive to case sufficient?
%
% Currently, |.add code:n| does not work cleanly with |.code:nx|. This
% is because the current method used to add to commands uses the stored
% body of the macro when complex arguments are present.  The handling
% here is better than \textsf{pgfkeys}, as no error is raised. However,
% a better method would be nice.  Ideas are welcome!
%
% The package makes it easy to create choices, but does not have an
% equivalent to \textsf{xkeyval}'s \cs{define@cmdkeys} or
% \cs{define@boolkeys} macros.  Is the extra ease of producing many
% keys at once worth the additional complexity?  If so, what would be
% suitable property names (suggestions |.multi set:n| and
% |.multi bool:n| come to mind).
%
% Does the package need to provide something like
% \cs{DisableKeyvalOption}? If so, does a method for globally disabling
% keys need to be provided?
%
%\subsection{Implementation}
%
% The usual preliminaries.  The key--value parsing itself is handled by
% \textsf{l3keyval}, which does the very low-level stuff so there is no
% need to worry here.
%    \begin{macrocode}
%<*package>
\ProvidesExplPackage
  {\filename}{\filedate}{\fileversion}{\filedescription}
\RequirePackage{l3keyval,l3messages,l3clist}
%    \end{macrocode}
%
%\subsubsection{Variables and contrasts}
%
%\begin{macro}{\c_keys_cs_prefix_tlp}
% First, the small number of constants needed are created. A prefix is
% used to keep all of the actual key macros in one place.
%    \begin{macrocode}
\tlp_new:Nn \c_keys_cs_prefix_tlp {keys}
%    \end{macrocode}
%\end{macro}
%\begin{macro}{\c_keys_root_tlp}
% The key root should have a clear name; like all of the key
% macros, this does not include the prefix.
%    \begin{macrocode}
\tlp_new:Nn \c_keys_root_tlp {/}
%    \end{macrocode}
%\end{macro}
%\begin{macro}{\c_keys_errors_path_tlp}
%\begin{macro}{\c_keys_properties_path_tlp}
%\begin{macro}{\c_keys_utilities_path_tlp}
% The locations of all of the keys used by \textsf{keys3}
% itself.
%    \begin{macrocode}
\tlp_new:Nn \c_keys_errors_path_tlp {/keys/errors}
\tlp_new:Nn \c_keys_properties_path_tlp {/keys/properties}
\tlp_new:Nn \c_keys_utilities_path_tlp {/keys}
%    \end{macrocode}
%\end{macro}
%\end{macro}
%\end{macro}
%
% The package starts by creating a number of storage areas used during
% key management.  Some of the typing here is problematic, as the
% |value| of a key can contain arbitrary code including |#| characters.
%  As far as possible, the key names (which should be strings) are
% stored in tlps, whereas the values are stored in toks.  There are
% some places where the values have to be transferred out: a dedicated
% tlp is reserved for this purpose.
%
%\begin{macro}{\l_keys_current_path_tlp}
%\begin{macro}{\l_keys_default_path_tlp}
%\begin{macro}{\l_keys_choice_path_tlp}
% The current and default paths can be stored as tlps. The default path
% is then initialised as the key root.
%    \begin{macrocode}
\tlp_new:N \l_keys_current_path_tlp
\tlp_new:N \l_keys_default_path_tlp
\tlp_set_eq:NN \l_keys_default_path_tlp \c_keys_root_tlp
\tlp_new:N \l_keys_choice_path_tlp
%    \end{macrocode}
%\end{macro}
%\end{macro}
%\end{macro}
%\begin{macro}{\l_keys_current_key_tlp}
%\begin{macro}{\l_keys_current_name_tlp}
% The ``key'' tlp holds the full key name including the path. The
% ``name'' tlp contains only the key name itself.
%    \begin{macrocode}
\tlp_new:N \l_keys_current_key_tlp
\tlp_new:N \l_keys_current_default_tlp
%    \end{macrocode}
%\end{macro}
%\end{macro}
%\begin{macro}{\l_keys_current_value_toks}
%\begin{macro}{\l_keys_current_value_tlp}
% For various reasons, a tlp is needed which points to the toks holding
% the current value.  Do not use the tlp!
%    \begin{macrocode}
\toks_new:N \l_keys_current_value_toks
\tlp_new:N \l_keys_current_value_tlp
%    \end{macrocode}
%\end{macro}
%\end{macro}
%\begin{macro}{\l_keys_current_module_tlp}
% The module name of the current module is stored here.
%    \begin{macrocode}
\tlp_new:N \l_keys_current_module_tlp
%    \end{macrocode}
%\end{macro}
%\begin{macro}{\l_keys_current_choice_tlp}
%\begin{macro}{\l_keys_current_choice_int}
% Multiple choices need some storage
%    \begin{macrocode}
\tlp_new:N \l_keys_current_choice_tlp
\int_new:N \l_keys_current_choice_int
%    \end{macrocode}
%\end{macro}
%\end{macro}
%\begin{macro}{\q_keys_stop}
%\begin{macro}{\q_value_required}
% Two new quarks are needed.
%    \begin{macrocode}
\quark_new:N \q_keys_stop
\quark_new:N \q_value_required
%    \end{macrocode}
%\end{macro}
%\end{macro}
%\begin{macro}{\l_keys_args_toks}
%\begin{macro}{\l_keys_body_toks}
% Storage areas used to add code to an existing key.
%    \begin{macrocode}
\toks_new:N \l_keys_args_toks
\toks_new:N \l_keys_body_toks
%    \end{macrocode}
%\end{macro}
%\end{macro}
%\begin{macro}{\l_keys_success_bool}
% A Boolean for trying keys.
%    \begin{macrocode}
\bool_new:N \l_keys_success_bool
%    \end{macrocode}
%\end{macro}
%\begin{macro}{\l_keys_tmpa_tlp}
% A scratch area.
%    \begin{macrocode}
\tlp_new:N \l_keys_tmpa_tlp
%    \end{macrocode}
%\end{macro}
%
%\subsubsection{Generally-available functions}
%
% A number of functions used internally to set keys are also part of
% the user interface for others to use.
%
%\begin{macro}{\quark_if_value_required:NTF}
%\begin{macro}{\quark_if_value_required:NT}
%\begin{macro}{\quark_if_value_required:NF}
% The quark test is the most basic function here: this is is exactly
% what \textsf{l3quark} would do for functions of these names.
%    \begin{macrocode}
\def_long_test_function_new:npn {quark_if_value_required:N} #1 {
  \if_meaning:NN \q_value_required #1
}
%    \end{macrocode}
%\end{macro}
%\end{macro}
%\end{macro}
%\begin{macro}{\keys_if_really_exist:cTF}
%\begin{macro}{\keys_if_really_exist:cT}
%\begin{macro}{\keys_if_really_exist:cF}
% Check if a key exists without adding to the hash table.
%    \begin{macrocode}
\def_long_test_function_new:npn {keys_if_really_exist:c} #1 {
  \if_cs_exist:w \c_keys_cs_prefix_tlp #1 \cs_end:
}
%    \end{macrocode}
%\end{macro}
%\end{macro}
%\end{macro}
%
% A number of new functions are needed that simple add the key root to
% the name provided and then use this as a csname for the underlying
% function.
%
%\begin{macro}{\keys_use:c}
%    \begin{macrocode}
\def_new:Npn \keys_use:c #1 {
  \use:c {\c_keys_cs_prefix_tlp #1}
}
%    \end{macrocode}
%\end{macro}
%\begin{macro}{\keys_def:cn}
%\begin{macro}{\keys_def:co}
%\begin{macro}{\keys_def:cx}
% Direct storage of data in keys.
%    \begin{macrocode}
\def_new:Npn \keys_def:cn #1#2 {
  \def:cpn {\c_keys_cs_prefix_tlp #1} {#2}
}
\def_new:Npn \keys_def:co #1#2 {
  \exp_args:Nno \keys_def:cn {#1} {#2}
}
\def_new:Npn \keys_def:cx #1#2 {
  \def:cpx {\c_keys_cs_prefix_tlp #1} {#2}
}
%    \end{macrocode}
%\end{macro}
%\end{macro}
%\end{macro}
%
%\subsubsection{The key parsing system}
%
%\begin{macro}{\keys_parse:n}
% The macro used to actually process the key--value input is taken from
% \textsf{l3keyval}.  There are two possible options, and so at this
% stage the macro is simply reserved.
%    \begin{macrocode}
\def_new:Npn \keys_parse:n #1 {\ERROR}
%    \end{macrocode}
%\end{macro}
%\begin{macro}{\keys_manage:n}
%\begin{macro}{\keys_manage_quick:n}
%\begin{macro}{\keys_manage_internal:n}
%\begin{macro}{\keys_manage_aux:nn}
% The main key management macros both call the auxiliary function after
% setting up the parser.  The expansion trick means a literal path is
% sent to the later function, and so the default path can be redefined.
%    \begin{macrocode}
\def_new:Npn \keys_manage:n #1 {
  \let:NN \keys_parse:n \KV_parse_space_removal_sanitize:n
  \tlp_clear:N \l_keys_current_module_tlp
  \exp_args:No \keys_manage_aux:nn {\l_keys_default_path_tlp} {#1}
}
\def_new:Npn \keys_manage_quick:n #1 {
  \let:NN \keys_parse:n \KV_parse_no_space_removal_no_sanitize:n
  \tlp_clear:N \l_keys_current_module_tlp
  \exp_args:No \keys_manage_aux:nn {\l_keys_default_path_tlp} {#1}
}
\def_new:Npn \keys_manage_internal:n #1 {
  \exp_args:No \keys_manage_aux:nn {\l_keys_default_path_tlp} {#1}
}
\def_new:Npn \keys_manage_aux:nn #1#2 {
  \tlp_set:Nn \l_keys_default_path_tlp {\c_keys_root_tlp}
  \keys_process:n {#2}
  \tlp_set:Nn \l_keys_default_path_tlp {#1}
}
%    \end{macrocode}
%\end{macro}
%\end{macro}
%\end{macro}
%\end{macro}
%\begin{macro}{\keys_manage:nn}
%\begin{macro}{\keys_manage_quick:nn}
%\begin{macro}{\keys_manage_aux:nnn}
% This version uses the same tricks but includes the path as a second
% argument.  When setting lots of keys separately, this is a little
% faster than the key-based method.
%    \begin{macrocode}
\def_new:Npn \keys_manage:nn #1#2 {
  \let:NN \keys_parse:n \KV_parse_space_removal_sanitize:n
  \exp_args:No \keys_manage_aux:nnn {\l_keys_default_path_tlp} {#1}
    {#2}
}
\def_new:Npn \keys_manage_quick:nn #1#2 {
  \let:NN \keys_parse:n \KV_parse_space_no_removal_no_sanitize:n
  \exp_args:No \keys_manage_aux:nnn {\l_keys_default_path_tlp} {#1}
    {#2}
}
\def_new:Npn \keys_manage_aux:nnn #1#2#3 {
  \tlp_set:Nn \l_keys_default_path_tlp {#2}
  \tlp_clear:N \l_keys_current_module_tlp
  \keys_process:n {#3}
  \tlp_set:Nn \l_keys_default_path_tlp {#1}
}
%    \end{macrocode}
%\end{macro}
%\end{macro}
%\end{macro}
%\begin{macro}{\keys_process:n}
% All of the management macros call this common parser. First, the key
% processing macros are defined, then the appropriate parser is called.
%    \begin{macrocode}
\def_new:Npn \keys_process:n #1{
  \let:NN \KV_key_value_elt:nn \keys_process_elt:nn
  \let:NN \KV_key_no_value_elt:n \keys_process_elt:n
  \keys_parse:n {#1}
}
%    \end{macrocode}
%\end{macro}
%\begin{macro}{\keys_process_elt:n}
% To keep things simple, the processor here simply uses the quark
% \cs{q_no_value} in place of the value and calls the standard
% function.
%    \begin{macrocode}
\def_new:Npn \keys_process_elt:n #1 {
  \keys_process_elt:nn {#1} {\q_no_value}
}
%    \end{macrocode}
%\end{macro}
%\begin{macro}{\keys_process_elt:nn}
%\begin{macro}{\keys_process_elt:}
% This is the engine room of the entire package. Each key is examined
% here, and after possibly adding a path a number of properties are
% checked for. First the package checks dedicated keys, then generic
% handlers.
%    \begin{macrocode}
\tlp_new:Nn \c_keys_no_value_tlp {\c_keys_really_no_value_tlp}
\tlp_new:N \c_keys_really_no_value_tlp
\def_new:Npn \keys_process_elt:nn #1#2 {
%    \end{macrocode}
% Using |:Nx| here ensures that any nesting does not result in the tlp
% containing non-text values.
%    \begin{macrocode}
  \tlp_set:Nx \l_keys_current_key_tlp {#1}
  \exp_after:NN \keys_path_add:w \l_keys_current_key_tlp \q_stop
  \toks_set:Nn \l_keys_current_value_toks {#2}
%    \end{macrocode}
% For the quark test, the ``funny'' tlp needs to be set.
%    \begin{macrocode}
  \def:Npx \l_keys_current_value_tlp
    {\toks_use:N \l_keys_current_value_toks}
  \quark_if_no_value:NT \l_keys_current_value_tlp {
    \keys_if_really_exist:cT {\l_keys_current_key_tlp /._default_tlp} {
      \keys_toks_set:Nc \l_keys_current_value_toks
        {\l_keys_current_key_tlp /._default_tlp}
%    \end{macrocode}
% There is a second test to come, so the odd tlp needs to be
% regenerated.
%    \begin{macrocode}
      \def:Npx \l_keys_current_value_tlp
        {\toks_use:N \l_keys_current_value_toks}
    }
  }
  \quark_if_value_required:NTF \l_keys_current_value_tlp {
    \keys_err_use:cw {value_required} \l_keys_current_key_tlp
  }{
    \keys_if_really_exist:cTF
      {\l_keys_current_key_tlp /._forbidden_bool} {
      \keys_err_use:cw {value_forbidden} \l_keys_current_key_tlp
    }{
      \keys_process_elt:
    }
  }
}
%    \end{macrocode}
% The key--value combination is at least valid for processing. The
% first real test is to look for a code handling function for the key.
% To keep things readable, this is a separate function.
%    \begin{macrocode}
\def_new:Npn \keys_process_elt: {
  \keys_if_cmd_really_exist:cTF {\l_keys_current_key_tlp} {
    \keys_use_cmd:c {\l_keys_current_key_tlp}
  }{
%    \end{macrocode}
% Next a check for the key name being defined. If it is, a second check
% is made to see if any value is available. If so, the key value is
% used; in the other case, the current value is saved to the key.
%    \begin{macrocode}
    \keys_if_really_exist:cTF {\l_keys_current_key_tlp}{
      \toks_if_empty:NTF \l_keys_current_value_toks {
        \keys_use:c {\l_keys_current_key_tlp}
      }{
        \quark_if_no_value:NTF \l_keys_current_value_tlp {
          \keys_use:c {\l_keys_current_key_tlp}
        }{
          \keys_def:co {\l_keys_current_key_tlp}
            {\toks_use:N \l_keys_current_value_toks}
        }
      }
    }{
%    \end{macrocode}
% If nothing has been found so far, the default properties need to be
% examined.  First, the path and name of the key must be separated.
% Generic property lists are then checked.
%    \begin{macrocode}
      \keys_separate:
      \keys_if_cmd_really_exist:cTF {
        \c_keys_properties_path_tlp / \l_keys_current_name_tlp
        }{
          \keys_use_cmd:c
            {\c_keys_properties_path_tlp / \l_keys_current_name_tlp}
        }{
%    \end{macrocode}
% Still no luck, and so the |unknown| paths are tried.
%    \begin{macrocode}
          \keys_if_cmd_really_exist:cTF {
            \l_keys_current_path_tlp /unknown
          }{
            \keys_use_cmd:c
              {\l_keys_current_path_tlp /unknown}
          }{
%    \end{macrocode}
% Nothing: call error message.
%    \begin{macrocode}
            \keys_err_use:cw {unknown_key} \l_keys_current_key_tlp
          }
        }
    }
  }
}
%    \end{macrocode}
%\end{macro}
%\end{macro}
%
%\subsubsection{Internal functions}
%
%\begin{macro}{\keys_path_add:w}
%\begin{macro}{\keys_path_add_aux:w}
% The code to check for a path looks for a |/| at the start of the key.
%    \begin{macrocode}
\def_new:Npn \keys_path_add:w {
  \exp_after:NN \peek_meaning:NTF \c_keys_root_tlp {
    \use_none_delimit_by_q_stop:w
  }{
    \keys_path_add_aux:w
  }
}
\def_new:Npn \keys_path_add_aux:w #1 \q_stop {
  \tlp_set:Nx \l_keys_current_key_tlp {\l_keys_default_path_tlp #1}
}
%    \end{macrocode}
%\end{macro}
%\end{macro}
%\begin{macro}{\keys_bool_new:c}
% Boolean value are always true here!
%    \begin{macrocode}
\def_new:Npn \keys_bool_new:c #1 {
  \bool_new:c {\c_keys_cs_prefix_tlp #1}
  \bool_set_true:c {\c_keys_cs_prefix_tlp #1}
}
%    \end{macrocode}
%\end{macro}
%\begin{macro}{\keys_choice_create:n}
% The system to create choices is quite simple.
%    \begin{macrocode}
\def_new:Npn \keys_choice_create:n #1 {
  \int_incr:N \l_keys_current_choice_int
  \keys_process:n {
    \l_keys_choice_path_tlp /#1/.code:x = {
      \exp_not:n {\int_set:Nn \l_keys_current_choice_int}
        {\int_use:N \l_keys_current_choice_int}
      \exp_not:n {\tlp_set:Nn \l_keys_current_choice_tlp} {#1}
      \toks_use:N \l_keys_body_toks
    }
  }
}
%    \end{macrocode}
%\end{macro}
%\begin{macro}{\keys_clear_properties:c}
% To avoid problem on redefinition, all properties are removed.
%    \begin{macrocode}
\def_new:Npn \keys_clear_properties:c #1 {
  \let:cN {\c_keys_cs_prefix_tlp #1/._args_toks} \c_undefined
  \let:cN {\c_keys_cs_prefix_tlp #1/._body_toks} \c_undefined
  \let:cN {\c_keys_cs_prefix_tlp #1/._boolean} \c_undefined
  \let:cN {\c_keys_cs_prefix_tlp #1/._cmd:w} \c_undefined
  \let:cN {\c_keys_cs_prefix_tlp #1/._default_tlp} \c_undefined
  \let:cN {\c_keys_cs_prefix_tlp #1/._forbidden_bool} \c_undefined
}
%    \end{macrocode}
%\end{macro}
%\begin{macro}{\keys_def_cmd:cn}
%\begin{macro}{\keys_def_cmd:cx}
%\begin{macro}{\keys_def_cmd:cnn}
%\begin{macro}{\keys_def_cmd:cnx}
% Creating the |._cmd:w| keys happens here.
%    \begin{macrocode}
\def_new:Npn \keys_def_cmd:cn #1#2 {
  \keys_clear_properties:c {#1}
  \def:cpn {\c_keys_cs_prefix_tlp #1/._cmd:w} ##1 \q_keys_stop {#2}
}
\def_new:Npn \keys_def_cmd:cx #1#2 {
  \keys_clear_properties:c {#1}
  \def:cpx {\c_keys_cs_prefix_tlp #1/._cmd:w} ##1 \q_keys_stop {#2}
}
\def_new:Npn \keys_def_cmd:cnn #1#2#3 {
  \keys_clear_properties:c {#1}
  \def:cpn {\c_keys_cs_prefix_tlp #1/._cmd:w} #2 \q_keys_stop {#3}
  \keys_toks_new:cn {#1/._args_toks} {#2}
  \keys_toks_new:cn {#1/._body_toks} {#3}
}
\def_new:Npn \keys_def_cmd:cnx #1#2#3 {
  \keys_clear_properties:c {#1}
  \def:cpx {\c_keys_cs_prefix_tlp #1/._cmd:w} #2 \q_keys_stop {#3}
  \keys_toks_new:cn {#1/._args_toks} {#2}
  \keys_toks_new:cn {#1/._body_toks} {
    \exp_not:n{\keys_err_use:cw {def_x_args} \l_keys_current_key_tlp}
    #3
  }
}
%    \end{macrocode}
%\end{macro}
%\end{macro}
%\end{macro}
%\end{macro}
%\begin{macro}{\keys_if_cmd_really_exist:cTF}
%\begin{macro}{\keys_if_cmd_really_exist:cT}
%\begin{macro}{\keys_if_cmd_really_exist:cF}
% A more specific version of an earlier test.
%    \begin{macrocode}
\def_long_test_function_new:npn {keys_if_cmd_really_exist:c} #1 {
  \if_cs_exist:w \c_keys_cs_prefix_tlp #1 /._cmd:w \cs_end:
}
%    \end{macrocode}
%\end{macro}
%\end{macro}
%\end{macro}
%\begin{macro}{\keys_expand:n}
% A helper for expansion.
%    \begin{macrocode}
\def_new:Npn \keys_expand:n #1 {
  \keys_process:n {\l_keys_current_path_tlp = #1}
}
%    \end{macrocode}
%\end{macro}
%\begin{macro}{\keys_let:cc}
%    \begin{macrocode}
\def_new:Npn \keys_let:cc #1#2 {
  \let:cc {\c_keys_cs_prefix_tlp #1} {\c_keys_cs_prefix_tlp #2}
}
%    \end{macrocode}
%\end{macro}
%\begin{macro}{\keys_module_name:}
% The module name is needed in a few places.
%    \begin{macrocode}
\def_new:Npn \keys_module_name: {
  \tlp_if_empty:NTF \l_keys_current_module_tlp {
    \tlp_clear:N \l_keys_tmpa_tlp
  }{
    \tlp_set:Nx \l_keys_tmpa_tlp {\l_keys_current_module_tlp _}
  }
}
%    \end{macrocode}
%\end{macro}
%\begin{macro}{\keys_put:ncn}
%    \begin{macrocode}
\def_new:Npn \keys_put:ncn #1#2#3 {
  \use:c {tlp_put_#1:cn} {\c_keys_cs_prefix_tlp #2} {#3}
}
%    \end{macrocode}
%\end{macro}
%\begin{macro}{\keys_put_code:nc}
% A function to add code to an existing key function.  This is a rather
% complex operation. If the key does not already exist, life is easy.
% On the other hand, if the key has already been defined, the existing
% definition is ``recovered'' to be patched. For keys with more than a
% single argument, |._args_toks| contains the appropriate pattern with
% the body in |._body_toks|.
%    \begin{macrocode}
\def_new:Npn \keys_put_code:nc #1#2 {
  \keys_if_really_exist:cTF {\l_keys_current_path_tlp /._cmd:w}{
    \keys_if_really_exist:cTF
      {\l_keys_current_path_tlp /._args_toks}{
      \toks_set_eq:Nc \l_keys_args_toks
        {\c_keys_cs_prefix_tlp \l_keys_current_path_tlp /._args_toks}
      \toks_set_eq:Nc \l_keys_body_toks
        {\c_keys_cs_prefix_tlp \l_keys_current_path_tlp /._body_toks}
    }{
%    \end{macrocode}
% With only one argument, the definition can be recovered out of the
% function itself.
%    \begin{macrocode}
      \toks_set:Nn \l_keys_args_toks {##1}
      \toks_set:Nn \l_keys_body_toks \exp_after:NN \exp_after:NN
        \exp_after:NN {\cs:w \c_keys_cs_prefix_tlp
        \l_keys_current_path_tlp /._cmd:w \cs_end: ##1 \q_keys_stop}
    }
%    \end{macrocode}
% The new function can now be built; this is the only point where the
% position of addition is important.
%    \begin{macrocode}
    \use:c {toks_put_#2:Nn} \l_keys_body_toks {#1}
    \tlp_set:Nx \l_keys_tmpa_tlp {\toks_use:N \l_keys_args_toks}
    \exp_args:NNOo \keys_def_cmd:cnx {\l_keys_current_path_tlp}
      \l_keys_tmpa_tlp {\toks_use:N \l_keys_body_toks}
  }{
    \keys_def_cmd:cn {\l_keys_current_path_tlp} {#1}
  }
}
%    \end{macrocode}
%\end{macro}
%\begin{macro}{\keys_separate:}
%\begin{macro}{\keys_separate:w}
% A simple piece of recursion.
%    \begin{macrocode}
\def_new:Npn \keys_separate: {
  \tlp_clear:N \l_keys_current_path_tlp
  \exp_after:NN \keys_separate:w \l_keys_current_key_tlp / \q_stop
}
\def_new:Npn \keys_separate:w /#1/#2 \q_stop {
  \tlist_if_empty:nTF {#2}{
    \tlp_set:Nn \l_keys_current_name_tlp {#1}
  }{
     \tlp_put_right:Nn \l_keys_current_path_tlp {/#1}
     \keys_separate:w /#2 \q_stop
  }
}
%    \end{macrocode}
%\end{macro}
%\end{macro}
%\begin{macro}{\keys_store:NN}
%\begin{macro}{\keys_store:Nc}
% Generic storage functions.
%    \begin{macrocode}
\def_new:Npn \keys_store:NN #1#2 {
  \keys_process:n {
    \c_keys_properties_path_tlp /.code:n = {
      #1 #2 {##1}
    }
  }
}
\def_new:Npn \keys_store:Nc #1#2 {
  \keys_module_name:
  \keys_process:n {
    \c_keys_properties_path_tlp /.code:x = {
      \exp_not:N #1 {l_ \l_keys_tmpa_tlp #2_tlp} {##1}
    }
  }
}
%    \end{macrocode}
%\end{macro}
%\end{macro}
%\begin{macro}{\keys_toks_new:cn}
%    \begin{macrocode}
\def_new:Npn \keys_toks_new:cn #1#2 {
  \toks_new:c {\c_keys_cs_prefix_tlp #1}
  \toks_set:cn {\c_keys_cs_prefix_tlp #1} {#2}
}
%    \end{macrocode}
%\end{macro}
%\begin{macro}{\keys_toks_set:Nc}
% Sets a toks to a key: not really intended for others to use. The
% contents of the key should be expandable, or protected with
% \cs{exp_not:E} or \cs{exp_not:n} otherwise (see for example the
% definition of |.value forbidden:|.
%    \begin{macrocode}
\def_new:Npn \keys_toks_set:Nc #1#2 {
  \toks_set:Nx #1 {\keys_use:c {#2}}
}
%    \end{macrocode}
%\end{macro}
%\begin{macro}{\keys_use_cmd:cn}
%\begin{macro}{\keys_use_cmd:c}
% Command keys are stored in a slightly odd fashion, and so this macro
% uses them in a ``safe'' way.
%    \begin{macrocode}
\def_new:Npn \keys_use_cmd:c #1 {
  \exp_args:Nno \keys_use_cmd:cn {#1}
    {\toks_use:N \l_keys_current_value_toks}
}
\def_new:Npn \keys_use_cmd:cn #1#2 {
  \quark_if_no_value:NTF \l_keys_current_value_tlp {
    \keys_use:c {#1/._cmd:w} \q_keys_stop
  }{
    \keys_use:c {#1/._cmd:w} #2 \q_keys_stop
  }
}
%    \end{macrocode}
%\end{macro}
%\end{macro}
%\begin{macro}{\keys_try:}
% Trying something without errors.
%    \begin{macrocode}
\def_new:Npn \keys_try: {
  \keys_if_really_exist:cTF {\l_keys_current_path_tlp /._cmd:w} {
    \quark_if_no_value:NT \l_keys_current_value_tlp {
      \keys_if_really_exist:cT
        {\l_keys_current_path_tlp /._default_tlp} {
        \keys_toks_set:Nc \l_keys_current_value_toks
          {\l_keys_current_path_tlp /._default_tlp}
      }
    }
    \let:NN \l_keys_current_key_tlp \l_keys_current_path_tlp 
    \keys_use_cmd:c {\l_keys_current_path_tlp}
    \bool_set_true:N \l_keys_success_bool
  }{
    \keys_if_really_exist:cTF {\l_keys_current_path_tlp} {
      \quark_if_no_value:NTF \l_keys_current_value_tlp {
        \keys_use:c {\l_keys_current_path_tlp}
      }{
        \tlp_set:co {\c_keys_cs_prefix_tlp \l_keys_current_path_tlp}
          {\toks_use:N \l_keys_current_value_toks}
      }
      \bool_set_true:N \l_keys_success_bool
    }{
      \bool_set_false:N \l_keys_success_bool
    }
  }
}
%    \end{macrocode}
%\end{macro}
%
%\begin{macro}{\keys_err_new:cNnnn}
% To create new error messages, a utility function is handy.
%    \begin{macrocode}
\def_new:Npn \keys_err_new:cNnnn #1 {
  \tlp_new:cn {l_keys_err_#1_tlp} {#1}
  \exp_args:NNc \err_interrupt_new:NNNnnn \c_keys_err_tlp
    {l_keys_err_#1_tlp}
}
%    \end{macrocode}
%\end{macro}
%\begin{macro}{\l_keys_err_unknown_key_tlp}
%\begin{macro}{\l_keys_err_value_ignored_tlp}
%\begin{macro}{\l_keys_err_value_required_tlp}
%\begin{macro}{\l_keys_err_def_x_args_tlp}
%\begin{macro}{\l_keys_err_boolean_expected_tlp}
%\begin{macro}{\l_keys_err_not_boolean_tlp}
%\begin{macro}{\l_keys_err_unknown_choice_tlp}
% Text for package error messages is stored in |keys3.err|. The error
% names do not need \cs{keys} at the start as they have to be in the
% file.
%    \begin{macrocode}
\err_file_new:Nn \c_keys_err_tlp {keys3.err}
\keys_err_new:cNnnn {unknown_key} 1
  {The~key~`#1'~is~unknown~and~is~being~ignored}
  {\err_help_return_or_X:}
  {}
\keys_err_new:cNnnn {value_required} 1
  {The~key~`#1'~requires~a~value~and~is~being~ignored}
  {\err_help_return_or_X:}
  {}
\keys_err_new:cNnnn {value_forbidden} 1
  {The~key~`#1'~cannot~taken~a~value:~\iow_newline:~\text_put_sp:~
   \text_put_sp:~the~one~given~is~being~ignored}
  {\err_help_return_or_X:}
  {}
\keys_err_new:cNnnn {def_x_args} 1
  {Key~`#1'~has~multiple~arguments~and~was~expanded:~\iow_newline:~
   \text_put_sp:~\text_put_sp:~adding~code~may~give~incorrect~results}
  {\err_help_return_or_X:}
  {}
\keys_err_new:cNnnn {boolean_expected} 2
  {Key~`#1'~takes~the~Boolean~values~`true'~and~`false'~\iow_newline:~
   \text_put_sp:~\text_put_sp:~only.~The~given~value~`#2'~is~ignored}
  {\err_help_return_or_X:}
  {}
\keys_err_new:cNnnn {not_boolean} 1
  {Key~`#1'~is~not~a~Boolean~key:~you~cannot~create~a~complement}
  {\err_help_return_or_X:}
  {}
\keys_err_new:cNnnn {unknown_choice} 2
  {Choice~`#2'~unknown~for~key~`#1':~\iow_newline:~\text_put_sp:~
   \text_put_sp:~the~value~give~is~being~ignored}
  {\err_help_return_or_X:}
  {}
\err_file_close:N \c_keys_err_tlp
%    \end{macrocode}
%\end{macro}
%\end{macro}
%\end{macro}
%\end{macro}
%\end{macro}
%\end{macro}
%\end{macro}
%\begin{macro}{\keys_err_use:cw}
% A second utility to use errors.
%    \begin{macrocode}
\def_new:Npn \keys_err_use:cw #1 {
  \exp_args:NNc \err_interrupt:NNw \c_keys_err_tlp {l_keys_err_#1_tlp}
}
%    \end{macrocode}
%\end{macro}
%
%\subsubsection{Property definitions}
%
% The |.code:n| and |.code:nn| properties have to be defined directly.
%    \begin{macrocode}
\keys_def_cmd:cn {\c_keys_properties_path_tlp /.code:n} {
  \keys_def_cmd:cn {\l_keys_current_path_tlp} {#1}
}
\keys_def_cmd:cnn {\c_keys_properties_path_tlp /.code:nn} {#1#2} {
  \keys_def_cmd:cnn {\l_keys_current_path_tlp} {#1} {#2}
}
%    \end{macrocode}
%
% The remaining definitions can all be carried out using the package
% itself. Here, the |_quick| method is used as there is no need to
% worry about spaces or catcodes.
%    \begin{macrocode}
\keys_manage_quick:n {
%    \end{macrocode}
% First, an error is created for unknown keys: this should help to
% prevent any odd internal errors.
%    \begin{macrocode}
  \c_keys_errors_path_tlp /unknown/.code:n = {
    \keys_err_use:cw {unknown_key} \l_keys_current_key_tlp
  },
%    \end{macrocode}
% The two |:x| definition methods need to be created.
%    \begin{macrocode}
  \c_keys_properties_path_tlp /.code:x/.code:n = {
    \keys_def_cmd:cx {\l_keys_current_path_tlp} {#1}
  },
  \c_keys_properties_path_tlp /.code:nx/.code:nn = {#1#2} {
    \keys_def_cmd:cnx {\l_keys_current_path_tlp} {#1} {#2}
  },
%    \end{macrocode}
% The package itself will need to be able to add code to an existing
% key, so the appropriate mechanism is created.
%    \begin{macrocode}
  \c_keys_properties_path_tlp /.put~code~right:n/.code:n = {
    \keys_put_code:nc {#1} {right}
  },
  \c_keys_properties_path_tlp /.put~code~left:n/.code:n = {
    \keys_put_code:nc {#1} {left}
  },
%    \end{macrocode}
% Initial and default values can now be created.
%    \begin{macrocode}
  \c_keys_properties_path_tlp /.initial:n/.code:n = {
    \keys_def:cn {\l_keys_current_path_tlp} {#1}
  },
  \c_keys_properties_path_tlp /.default:n/.code:n = {
    \keys_def:cn {\l_keys_current_path_tlp /._default_tlp} {#1}
  },
%    \end{macrocode}
% Values are either required or forbidden, with the appropriate error
% messages. The \cs{q_value_required} quark must no be expanded but
% must work in the quark test. Hence the expansion protect used here:
% remember this is stored in a tlp!
%    \begin{macrocode}
  \c_keys_properties_path_tlp /.value~required:/.code:n = {
    \keys_def:cn {\l_keys_current_path_tlp /._default_tlp}
      {\exp_not:E \q_value_required}
  },
  \c_keys_errors_path_tlp /value~required/.code:n = {
    \keys_err_use:cw {value_required} \l_keys_current_key_tlp
  },
  \c_keys_properties_path_tlp /.value~forbidden:/.code:n = {
    \keys_bool_new:c {\l_keys_current_path_tlp /._forbidden_bool}
  },
  \c_keys_errors_path_tlp /value~forbidden/.code:n = {
    \keys_err_use:cw {value_forbidden} \l_keys_current_key_tlp
  },
%    \end{macrocode}
% Being able to change path is pretty useful.
%    \begin{macrocode}
  \c_keys_properties_path_tlp /.cd:/.code:n = {
    \tlp_set:Nx \l_keys_default_path_tlp {\l_keys_current_path_tlp /}
  },
  \c_keys_properties_path_tlp /.module:/.code:n = {
    \keys_manage_internal:n {
      \l_keys_current_path_tlp /.code:x = {
        \exp_not:N \tlp_set:Nx \exp_not:N \l_keys_default_path_tlp
          {\l_keys_current_path_tlp /}
      }
    }
  },
%    \end{macrocode}
% Pre-built lists of keys can be created for regular use.
%    \begin{macrocode}
  \c_keys_properties_path_tlp /.apply~keys:n/.code:n = {
    \keys_manage_internal:n {
      \l_keys_current_path_tlp /.code:n = {\keys_process:n {#1}}
    }
  },
  \c_keys_properties_path_tlp /.apply~keys:x/.code:n = {
    \keys_manage_internal:n {
      \l_keys_current_path_tlp /.code:x = {
        \exp_not:N \keys_process:n {#1}
      }
    }
  },
  \c_keys_properties_path_tlp /.apply~keys:nn/.code:nn = {#1#2} {
    \keys_manage_internal:n {
      \l_keys_current_path_tlp /.code:nn = {#1} {\keys_process:n {#2}}
    }
  },
  \c_keys_properties_path_tlp /.apply~keys:nx/.code:nn = {#1#2} {
    \keys_manage_internal:n {
      \l_keys_current_path_tlp /.code:x =
        {#1} {\exp_not:N \keys_process:n {#2}}
    }
  },
%    \end{macrocode}
% Code can also be added to the above.
%    \begin{macrocode}
  \c_keys_properties_path_tlp /.put~keys~left:n/.code:n = {
    \keys_manage_internal:n {
      \l_keys_current_path_tlp /.put~code~left:n =
        {\keys_process:n {#1}}
    }
  },
  \c_keys_properties_path_tlp /.put~keys~right:n/.code:n = {
    \keys_manage_internal:n {
      \l_keys_current_path_tlp /.put~code~right:n =
        {\keys_process:n {#1}}
    }
  },
%    \end{macrocode}
% Transfer of data in and out.
%    \begin{macrocode}
  \c_keys_properties_path_tlp /.key~to~tlp:N/.code:n = {
    \tlp_set_eq:Nc #1 {\c_keys_cs_prefix_tlp \l_keys_current_path_tlp}
  },
  \c_keys_properties_path_tlp /.key~to~tlp:c/.code:n = {
    \tlp_set_eq:cc {#1} {\c_keys_cs_prefix_tlp \l_keys_current_path_tlp}
  },
  \c_keys_properties_path_tlp /.tlp~to~key:N/.code:n = {
    \tlp_set_eq:cN {\c_keys_cs_prefix_tlp \l_keys_current_path_tlp} #1
  },
  \c_keys_properties_path_tlp /.tlp~to~key:c/.code:n = {
    \tlp_set_eq:cc {\c_keys_cs_prefix_tlp \l_keys_current_path_tlp} {#1}
  },
%    \end{macrocode}
% Adding to literal values is easier than adding to code.
%    \begin{macrocode}
  \c_keys_properties_path_tlp /.put~right:n/.code:n = {
    \keys_put:ncn {right} {\l_keys_current_path_tlp} {#1}
  },
  \c_keys_properties_path_tlp /.put~left:n/.code:n = {
    \keys_put:ncn {left} {\l_keys_current_path_tlp} {#1}
  },
%    \end{macrocode}
% That is all of the lower-level things done.  Now use them to build
% some higher-level functionality.
%    \begin{macrocode}
  \c_keys_properties_path_tlp /.store~in~tlp:N/.code:n = {
    \keys_store:NN \tlp_set:Nn #1
  },
  \c_keys_properties_path_tlp /.xstore~in~tlp:N/.code:n = {
    \keys_store:NN \tlp_set:Nx #1
  },
  \c_keys_properties_path_tlp /.store~in~tlp:c/.code:n = {
    \keys_store:Nc \tlp_set:cn {#1}
  },
  \c_keys_properties_path_tlp /.xstore~in~tlp:c/.code:n = {
    \keys_store:Nc \tlp_set:cx {#1}
  },
  \c_keys_properties_path_tlp /.store~in~int:N/.code:n = {
    \keys_store:NN \int_set:Nn #1
  },
  \c_keys_properties_path_tlp /.store~in~int:c/.code:n = {
    \keys_store:Nc \int_set:cn {#1}
  },
  \c_keys_properties_path_tlp /.store~in~skip:N/.code:n = {
    \keys_store:NN \skip_set:Nn #1
  },
  \c_keys_properties_path_tlp /.store~in~skip:c/.code:n = {
    \keys_store:Nc \skip_set:cn {#1}
  },
  \c_keys_properties_path_tlp /.store~in~toks:N/.code:n = {
    \keys_store:NN \toks_set:Nn #1
  },
  \c_keys_properties_path_tlp /.store~in~toks:c/.code:n = {
    \keys_store:Nc \toks_set:cn {#1}
  },
%    \end{macrocode}
% Global versions of all of the above storage properties.
%    \begin{macrocode}
  \c_keys_properties_path_tlp /.gstore~in~tlp:N/.code:n = {
    \keys_store:NN \tlp_gset:Nn #1
  },
  \c_keys_properties_path_tlp /.xgstore~in~tlp:N/.code:n = {
    \keys_store:NN \tlp_gset:Nx #1
  },
  \c_keys_properties_path_tlp /.gstore~in~tlp:c/.code:n = {
    \keys_store:Nc \tlp_gset:cn {#1}
  },
  \c_keys_properties_path_tlp /.xgstore~in~tlp:c/.code:n = {
    \keys_store:Nc \tlp_gset:cx {#1}
  },
  \c_keys_properties_path_tlp /.gstore~in~int:N/.code:n = {
    \keys_store:NN \int_gset:Nn #1
  },
  \c_keys_properties_path_tlp /.gstore~in~int:c/.code:n = {
    \keys_store:Nc \int_gset:cn {#1}
  },
  \c_keys_properties_path_tlp /.gstore~in~skip:N/.code:n = {
    \keys_store:NN \skip_gset:Nn #1
  },
  \c_keys_properties_path_tlp /.gstore~in~skip:c/.code:n = {
    \keys_store:Nc \skip_gset:cn {#1}
  },
  \c_keys_properties_path_tlp /.gstore~in~toks:N/.code:n = {
    \keys_store:NN \toks_gset:Nn #1
  },
  \c_keys_properties_path_tlp /.gstore~in~toks:c/.code:n = {
    \keys_store:Nc \toks_gset:cn {#1}
  },
%    \end{macrocode}
% Functions should not be stored in tlps.
%    \begin{macrocode}
  \c_keys_properties_path_tlp /.let~to~func:N/.code:n = {
    \keys_process:n {
      \c_keys_properties_path_tlp /.code:n = {
        \let:NN #1 ##1
      }
    }
  },
  \c_keys_properties_path_tlp /.let~to~func:c/.code:n = {
    \keys_module_name:
    \keys_process:n {
      \c_keys_properties_path_tlp /.code:x = {
        \exp_not:N \let:cN {\l_keys_tmpa_tlp #1:w} ##1
      }
    }
  },
%    \end{macrocode} 
% Boolean keys are created with a true default.
%    \begin{macrocode}
  \c_keys_properties_path_tlp /.boolean:N/.code:n = {
    \cs_if_free:NT #1 {\bool_new:N #1}
    \keys_process:n {
      \l_keys_current_path_tlp /.code:n = {
        \cs_if_really_exist:cTF {bool_set_##1:c} {
          \use:c {bool_set_##1:N} #1
        }{
          \keys_err_use:cw {boolean_expected} \l_keys_current_key_tlp
            {\toks_use:N \l_keys_current_value_toks}
        }
      },
      \l_keys_current_path_tlp /.default:n = true
    }
  \keys_def:cn {\l_keys_current_path_tlp /._boolean} {\exp_not:N #1}
  },
%    \end{macrocode}
% The |:c| variant has to create the appropriate name first.
%    \begin{macrocode}
  \c_keys_properties_path_tlp /.boolean:c/.code:n = {
    \keys_module_name:
    \cs_if_free:cT {l_ \l_keys_tmpa_tlp #1_bool} {
      \bool_new:c {l_ \l_keys_tmpa_tlp #1_bool}
    }
    \keys_process:n {
      \l_keys_current_path_tlp /.code:x = {
        \exp_not:N \cs_if_really_exist:cTF {bool_set_##1:N} {
          \exp_not:N \use:c {bool_set_##1:N}
            \exp_not:d {\use:c {l_ \l_keys_tmpa_tlp #1 _bool}}
        }{
          \exp_not:n {
            \keys_err_use:cw {boolean_expected}
              \l_keys_current_key_tlp
              {\toks_use:N \l_keys_current_value_toks}
          }
        }
      },
      \l_keys_current_path_tlp /.default:n = true
    }
  \keys_def:cx {\l_keys_current_path_tlp /._boolean}
    {\exp_not:N \exp_not:N \exp_not:d
      {\use:c {l_ \l_keys_tmpa_tlp #1_bool}}}
  },
%    \end{macrocode}
% Once again global versions.
%    \begin{macrocode}
  \c_keys_properties_path_tlp /.gboolean:N/.code:n = {
    \cs_if_free:NT #1 {\bool_new:N #1}
    \keys_process:n {
      \l_keys_current_path_tlp /.code:n = {
        \cs_if_really_exist:cTF {bool_gset_##1:c} {
          \use:c {bool_gset_##1:N} #1
        }{
          \keys_err_use:cw {boolean_expected} \l_keys_current_key_tlp
            {\toks_use:N \l_keys_current_value_toks}
        }
      },
      \l_keys_current_path_tlp /.default:n = true
    }
  \keys_def:cn {\l_keys_current_path_tlp /._boolean} {\exp_not:N #1}
  },
  \c_keys_properties_path_tlp /.gboolean:c/.code:n = {
    \keys_module_name:
    \cs_if_free:cT {l_ \l_keys_tmpa_tlp #1_bool} {
      \bool_new:c {l_ \l_keys_tmpa_tlp #1_bool}
    }
    \keys_process:n {
      \l_keys_current_path_tlp /.code:x = {
        \exp_not:N \cs_if_really_exist:cTF {bool_gset_##1:N} {
          \exp_not:N \use:c {bool_gset_##1:N}
            \exp_not:d {\use:c {l_ \l_keys_tmpa_tlp #1 _bool}}
        }{
          \exp_not:n {
            \keys_err_use:cw {boolean_expected}
              \l_keys_current_key_tlp
              {\toks_use:N \l_keys_current_value_toks}
          }
        }
      },
      \l_keys_current_path_tlp /.default:n = true
    }
  \keys_def:cx {\l_keys_current_path_tlp /._boolean}
    {\exp_not:N \exp_not:N \exp_not:d
      {\use:c {l_ \l_keys_tmpa_tlp #1_bool}}}
  },
%    \end{macrocode}
% Creating an inverse needs a little work as the logic needs to be
% reversed.
%    \begin{macrocode}
  \c_keys_properties_path_tlp /.inverse~key:n/.code:n = {
    \keys_if_really_exist:cTF {\l_keys_current_path_tlp /._boolean} {
      \tlp_set_eq:NN \l_keys_tmpa_tlp \l_keys_current_path_tlp
      \tlp_set_eq:NN \l_keys_current_key_tlp \l_keys_current_path_tlp
      \keys_separate:
      \keys_process:n {
        \l_keys_current_path_tlp /#1/.code:x = {
          \exp_not:N \tlist_if_eq:nnTF {true} {##1} {
            \exp_not:N \bool_set_false:N \keys_use:c
              {\l_keys_tmpa_tlp /._boolean}
          }{
            \exp_not:N \tlist_if_eq:nnTF {false} {##1} {
              \exp_not:N \bool_set_true:N \keys_use:c
                {\l_keys_tmpa_tlp /._boolean}
            }{
              \exp_not:n {
                \keys_err_use:cw {boolean_expected}
                  \l_keys_current_key_tlp
                  {\toks_use:N \l_keys_current_value_toks}
              }
            }
          }
        },
        \l_keys_current_path_tlp /#1/.default:n = true
      }
    }{
      \keys_err_use:cw {not_boolean} \l_keys_current_path_tlp
    }
  },
%    \end{macrocode}
% Choices are handled in a slightly different way.
%    \begin{macrocode}
  \c_keys_properties_path_tlp /.choice:/.code:n = {
    \keys_manage_internal:n {
      \l_keys_current_path_tlp /.cd:,
      .code:n = {
        \tlp_set:Nn \l_keys_current_choice_tlp {##1}
        \int_zero:N \l_keys_current_choice_int
        \exp_args:No \keys_process:n  {\l_keys_current_key_tlp /##1}
      },
      unknown/.code:n = {
        \keys_err_use:cw {unknown_choice} \l_keys_current_path_tlp
          {\l_keys_current_name_tlp}
      }
    }
  },
%    \end{macrocode}
% Multiple choices can also be created.
%    \begin{macrocode}
  \c_keys_properties_path_tlp /.choices:nn/.code:nn = {#1#2} {
    \tlp_set_eq:NN \l_keys_choice_path_tlp \l_keys_current_path_tlp
    \toks_set:Nn \l_keys_body_toks {#2}
    \int_zero:N \l_keys_current_choice_int
    \clist_map_function:nN {#1} \keys_choice_create:n
    \keys_manage_internal:n {
      \l_keys_choice_path_tlp  /.choice:
    }
  },
%    \end{macrocode}
% Different names for the same thing.
%    \begin{macrocode}
  \c_keys_properties_path_tlp /.equivalent~key:n/.code:n = {
    \keys_if_really_exist:cT {\l_keys_current_path_tlp /._cmd:w} {
      \tlp_set_eq:NN \l_keys_current_key_tlp \l_keys_current_path_tlp
      \keys_separate:
      \keys_let:cc {\l_keys_current_path_tlp /#1/._cmd:w}
        {\l_keys_current_key_tlp /._cmd:w}
    }
  },
%    \end{macrocode}
% Expansion keys need to be available.
%    \begin{macrocode}
  \c_keys_properties_path_tlp /.expand:o/.code:n = {
    \exp_args:No \keys_expand:n {#1}
  },
  \c_keys_properties_path_tlp /.expand:d/.code:n = {
    \exp_args:NNo \exp_args:No \keys_expand:n {#1}
  },
  \c_keys_properties_path_tlp /.expand:x/.code:n = {
    \exp_args:Nx \keys_expand:n {#1}
  },
%    \end{macrocode}
% ``Trying'' a key.
%    \begin{macrocode}
  \c_keys_properties_path_tlp /.try:n/.code:n = {\keys_try:},
  \c_keys_properties_path_tlp /.retry:n/.code:n = {
    \bool_if:NF \l_keys_success_bool {\keys_try:}
  },
%    \end{macrocode}
%
% Some debugging keys.
%    \begin{macrocode}
  \c_keys_properties_path_tlp /.show~code:/.code:n = {
    \cs_show:c {\c_keys_cs_prefix_tlp \l_keys_current_path_tlp /._cmd:w}
  },
  \c_keys_properties_path_tlp /.show:/.code:n = {
    \cs_show:c {\c_keys_cs_prefix_tlp \l_keys_current_path_tlp}
  },
%    \end{macrocode}
% Keys for programmers.
%    \begin{macrocode}
  \c_keys_utilities_path_tlp /execute:n/.code:n = {#1},
  \c_keys_utilities_path_tlp /current~module:n/.store~in~tlp:N =
    \l_keys_current_module_tlp
}
%</package>
%    \end{macrocode}
%
%\Finale
