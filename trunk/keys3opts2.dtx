%\iffalse
%% File: keys3opts2.dtx Copyright (C) 2008 Joseph Wright
%%
%% It may be distributed and/or modified under the conditions of the
%% LaTeX Project Public License (LPPL), either version 1.3c of this
%% license or (at your option) any later version.  The latest version
%% of this license is in the file
%%
%%    http://www.latex-project.org/lppl.txt
%%
%% -----------------------------------------------------------------------
%<*driver|package>
\RequirePackage{l3names}
%</driver|package>
%\fi
\GetIdInfo$Id: keys3opts2.dtx 8 2008-09-25 06:11:19Z josephwright $
  {Using keys3 for LaTeX2e options}
%\iffalse
%<*driver>
%\fi
\ProvidesFile{\filename.\filenameext}
  [\filedate\space v\fileversion\space\filedescription]
%\iffalse
\documentclass[full]{l3doc}
\begin{document}
\DocInput{\filename.\filenameext}
\end{document}
%</driver>
% \fi
%
% \title{The \textsf{keys3opts2} package^^A
%   \thanks{This file has version number \fileversion, last
%     revised \filedate.}\\
% Using \textsf{keys3} for \LaTeXe\ options}
% \author{Joseph Wright^^A
%   \thanks{E-mail: joseph.wright@morningstar2.co.uk}}
% \date{\filedate}
% \maketitle
%
%\section{Package options}
%
% The methods that \LaTeX3 will use to process package options are not
% clear at the present time.  However, in order to fully test the
% \textsf{expl3} bundle fully, it is useful to be able to work with
% the \LaTeXe\ system but using \LaTeX3 methods as far as possible.
%
% The \textsf{keys3} package provides an experimental method for
% managing key--value input using \textsf{expl3} syntax. It is
% self-contained and does not use any \LaTeXe methods. On the other
% hand, \textsf{keys3opts2} is intended as a ``bridge'' between the
% \LaTeXe\ kernel and \textsf{keys3}. It allows keys to be created
% using \textsf{keys3} and processed by \LaTeXe\ as package or class
% options.
%
%\subsection{functions}
%
% \begin{function}{\keys_options_ltx:n}
%   \begin{syntax}
%     "\keys_options_ltx:n" \Arg{module}
%   \end{syntax}
%   The only macro for outside use. Processes the current option list
%   using the currently defined keys, checking if they exist for
%   <module>.
% \end{function}
%
% \begin{function}{%
%   \keys_options_ltx_cls:n|
%   \keys_options_ltx_pkg:n}
%   \begin{syntax}
%     "\keys_options_ltx_cls:n" \Arg{option}
%   \end{syntax}
%   Helper macros for processing class and package options,
%   respectively.
% \end{function}
%
% \begin{function}{\keys_options_ltx_strip_eq:w}
%   \begin{syntax}
%     "\keys_options_ltx_strip_eq:w" <option> "\q_stop"
%   \end{syntax}
%   Used to strip ``|=|'' out of options.
% \end{function}
%
% \begin{variable}{\l_keys_options_ltx_clist}
%   Storage area for option list.
% \end{variable}
%
%\subsection{Implementation}
%
% Almost everything here is based heavily on \textsf{pgfopts}, which is
% itself a port of the appropriate code from \textsf{kvoptions} to work
% with \textsf{pgfkeys}.
%
% As this package is an add-on to \textsf{keys3}, the same internal
% module name (|\keys_|) is used here as in the main package.
%
% The usual preliminaries.
%    \begin{macrocode}
%<*package>
\ProvidesExplPackage
  {\filename}{\filedate}{\fileversion}{\filedescription}
\RequirePackage{keys3}
%    \end{macrocode}
%
%\begin{macro}{\l_keys_options_ltx_clist}
%    \begin{macrocode}
\clist_new:N \l_keys_options_ltx_clist
%    \end{macrocode}
%\end{macro}
%
%\begin{macro}{\keys_options_ltx:n}
% As this is a minimal testing package, a lot of the complexity of the
% normal \LaTeX\ options is ignored. Here, the package insists a path
% is given for the keys. First, a path may need to be added to the
% argument given. The path we need to process is then stored in
% \cs{l_keys_current_key_tlp}.
%    \begin{macrocode}
\def_new:Npn \keys_options_ltx:n #1 {
  \tlp_set:Nx \l_keys_current_key_tlp {#1}
  \exp_after:NN \keys_path_add:w \l_keys_current_key_tlp \q_stop
  \clist_clear:N \l_keys_options_ltx_clist
%    \end{macrocode}
% Processing now depends on whether a class is being processed.
% If it is and there are options, only known ones should be processed
% as unknown ones are passed thought to packages.
%    \begin{macrocode}
  \cs_if_eq:NNTF \@currext \@clsextension {
    \cs_if_free:cF {opt@ \@currname . \@currext} {
      \clist_map_function:cN {opt@ \@currname . \@currext}
        \keys_options_ltx_cls:n
    }
  }{
%    \end{macrocode}
% When not processing a class, only the global options that are known
% to the package should be looked at.
%    \begin{macrocode}
    \cs_if_free:NF \@classoptionslist {
      \clist_map_function:NN \@classoptionslist \keys_options_ltx_pkg:n
    }
    \cs_if_free:cF {opt@ \@currname . \@currext} {
      \exp_args:NNc \clist_concat:NNN \l_keys_options_ltx_clist
        {opt@ \@currname . \@currext} \l_keys_options_ltx_clist
    }
  }
  \clist_put_left:Nn \l_keys_options_ltx_clist {#1/.cd:}
  \exp_args:No \keys_manage:n {\l_keys_options_ltx_clist}
  \AtEndOfPackage {\let:NN \@unprocessedoptions \scan_stop:}
}
%    \end{macrocode}
%\end{macro}
%\begin{macro}{\keys_options_ltx_cls:n}
% This macro checks if the current class option is a known key. To do
% that, any ``|=|'' part has to be removed first. The code key is
% checked for, hence the |/._cmd:w| part.
%    \begin{macrocode}
\def_new:Npn \keys_options_ltx_cls:n #1 {
  \keys_if_really_exist:cTF {
    \l_keys_current_key_tlp /
    \keys_options_ltx_strip_eq:w #1 = \q_stop
    /._cmd:w
  } {
    \clist_put_right:Nn \l_keys_options_ltx_clist {#1}
  }{
    \clist_put_right:Nn \@unusedoptionlist {#1}
  }
}
%    \end{macrocode}
%\end{macro}
%\begin{macro}{\keys_options_ltx_pkg:n}
% For packages, options need to be removed from the global list.
%    \begin{macrocode}
\def_new:Npn \keys_options_ltx_pkg:n #1 {
  \keys_if_really_exist:cT {
    \l_keys_current_key_tlp /
    \keys_options_ltx_strip_eq:w #1 = \q_stop
    /._cmd:w
  } {
    \clist_put_right:Nn \l_keys_options_ltx_clist {#1}
%    \end{macrocode}
% At this stage, some \LaTeXe\ is needed; the option is removed from
% the unused list.
%    \begin{macrocode}
    \exp_args:Nnx \@removeelement {#1} \@unusedoptionlist
      \@unusedoptionlist
  }
}
%    \end{macrocode}
%\end{macro}
%\begin{macro}{\keys_options_ltx_strip_eq:w}
%    \begin{macrocode}
\def_new:Npn \keys_options_ltx_strip_eq:w #1=#2 \q_stop {#1}
%    \end{macrocode}
%\end{macro}
%
%    \begin{macrocode}
%</package>
%    \end{macrocode}
%
%\Finale
